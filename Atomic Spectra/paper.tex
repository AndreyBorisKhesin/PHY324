\input{header}
\physics
\begin{document}
\papertitle{Analysis of Atomic Spectra of Various Gases}
\paperauth{A}{Khesin}{Lab Technician}
\paperauth{P}{Zavyalova}{Data Analyst}
\paperdate{January 15, 2018}
\begin{paperabs}
Abstract goes here
\end{paperabs}

\begin{paper}
\papersec{Introduction}

In this exercise, a prism spectrometer is calibrated and used to determine the Rydberg constant \( R_H \). Further, an ``unknown'' gas is identified by measuring its spectral lines and comparing them with the obtained calibration curve. 

\papersec{Method}

Method goes here

\papersec{Data}

(Helium)

\begin{tabular}{|l|c|r|}
	\hline
	Vernier & Wavelength (nm) & Color \\ \hline
	6.66 & 706.5 & red \\ \hline
	7.06 & 667.8 & red \\ \hline
	8.31 & 587.6 & yellow \\ \hline
	11.09 & 501.6 & green \\ \hline
	11.50 & 492.2 & green \\ \hline
	12.58 & 471.3 & blue \\ \hline
	14.19 & 447.1 & blue \\ \hline
\end{tabular}

(Hydrogen)

\begin{tabular}{|l|r|}
	\hline
	Vernier & Color \\ \hline
	7.26 & red \\ \hline
	11.80 & blue-green \\ \hline
	15.15 & violet \\ \hline
	17.75 & violet \\ \hline
\end{tabular}

(Unknown \# 7)

\begin{tabular}{|l|r|}
	\hline
	Vernier & Color \\ \hline
	8.44 & orange \\ \hline
	9.18 & green \\ \hline
	13.89 & violet \\ \hline
	15.38 & violet \\ \hline
\end{tabular}

\papersec{Analysis}

Analysis goes here

\papersec{Conclusion}

Conclusion goes here
\end{paper}
\end{document}