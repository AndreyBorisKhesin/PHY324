\input{header}
\physics
\begin{document}
\papertitle{Analysis of Atomic Spectra of Various Gases}
\paperauth{A}{Khesin}{Lab Technician}
\paperauth{P}{Zavyalova}{Data Analyst}
\paperdate{January 15, 2018}
\begin{paperabs}
TODO Abstract
\end{paperabs}

\begin{paper}
\papersec{Introduction}

The purpose of this experiment was to use a prism spectrometer to identify
unknown gases based on their atomic spectra.
Furthermore, the Rydberg constant ($R_H$) was determined by examining the
spectral lines of hydrogen.
The apparatus consisted of a spectrometer, a sodium lamp, a tube of hydrogen
gas, a tube of helium gas, and two tubes of unknown compounds marked ``Unknown
\#7'' and ``Unknown \#9''.

The spectrometer was calibrated by aligning the eyepiece with the lines produced
by the sodium lamp and the Hartmann relation method was used to determine the
relation between the observed wavelength and the scale reading on the
spectrometer.

By examining the spectral lines of hydrogen, the Rydberg constant was
determined with the following formula.

\papereq{Rydberg}{\frac1\lambda=R_H\pars{\frac14-\frac1{n^2}}}{Serbanescu}
\begin{paperwhere}
\papervar{\lambda}{observed wavelength}{\meter}
\papervar{R_H}{Rydberg constant}{\per\meter}
\papervar{n}{index of spectral line}{}
\end{paperwhere}

To determine the relation between the scale reading of the spectrometer and the
observed wavelength, the Hartmann relation was used.

\papereq{Hartmann}{y=\frac{m}{\lambda-\lambda_0}+b}{Serbanescu}
\begin{paperwhere}
\papervar{y}{scale reading}{}
\papervar{m}{slope of fit}{\meter}
\papervar{\lambda}{observed wavelength}{\meter}
\papervar{\lambda_0}{fundamental wavelength of spectrometer}{\meter}
\papervar{b}{constant of fit}{}
\end{paperwhere}

For the spectrometer used in the experiment, the value of $\lambda_0$ was
provided as $(285.2\pm.4)\si{\nano\meter}$.

The experiment was setup as shown.
\paperfig{Setup}{\pdf{setup}}{The experimental setup.
The spectrometer (left) is pointing at a tube containing ``Unknown \#7'' (right).}
\papersec{Observations}

The vernier scale on the spectrometer had a reading error of
$\pm.02\si{\nano\meter}$.
The uncertainties in the wavelengths for the spectral lines of helium were
deemed too small to consider.

The vernier scale readings were recorded for helium, hydrogen, ``Unknown \#7'',
and ``Unknown \#9''.
Additionally, the colours of the observed lines and matched wavelengths for the
spectral lines of helium were recorded.

\paperfig{Helium}{\begin{papertable}{|[outer]I|[inner]C|[inner]C|[inner]C|[outer]}
\paperoline\papertableindexheader&\textsc{Scale}&\textsc{Wavelength}&\multirow{2}{*}{\textsc{Colour}}\\
\papertablecheaderunit{.02}{\hspace{-0.5ex}}\papertablecheadersymbol{\lambda~(\si{\nano\meter})}&\\\paperiline
\papertableindex\papertablecval{6.66}\papertablecval{706.5}\papertablecval{Red}\\\paperiline
\papertableindex\papertablecval{7.06}\papertablecval{667.8}\papertablecval{Red}\\\paperiline
\papertableindex\papertablecval{8.31}\papertablecval{587.6}\papertablecval{Yellow}\\\paperiline
\papertableindex\papertablecval{11.09}\papertablecval{501.6}\papertablecval{Green}\\\paperiline
\papertableindex\papertablecval{11.50}\papertablecval{492.2}\papertablecval{Green}\\\paperiline
\papertableindex\papertablecval{12.58}\papertablecval{471.3}\papertablecval{Blue}\\\paperiline
\papertableindex\papertablecval{14.19}\papertablecval{447.1}\papertablecval{Blue}\\\paperoline
\end{papertable}\vspace{-1.5em}}
{The scale readings, matched wavelengths, and colours of the spectral lines of helium.}\vspace{1em}

\paperfig{Hydrogen}{\begin{papertable}{|[outer]I|[inner]C|[inner]C|[outer]}
\paperoline\papertableindexheader&\textsc{Scale}&\multirow{2}{*}{\textsc{Colour}}\\
\papertablecheaderunit{.02}{\hspace{-0.5ex}}&\\\paperiline
\papertableindex\papertablecval{7.26}\papertablecval{Red}\\\paperiline
\papertableindex\papertablecval{11.80}\papertablecval{Blue-Green}\\\paperiline
\papertableindex\papertablecval{15.15}\papertablecval{Violet}\\\paperiline
\papertableindex\papertablecval{17.75}\papertablecval{Violet}\\\paperoline
\end{papertable}\vspace{-1.5em}}
{The scale readings and colours of the spectral lines of hydrogen.}\vspace{1em}

\paperfig{UnknownSeven}{\begin{papertable}{|[outer]I|[inner]C|[inner]C|[outer]}
\paperoline\papertableindexheader&\textsc{Scale}&\multirow{2}{*}{\textsc{Colour}}\\
\papertablecheaderunit{.02}{\hspace{-0.5ex}}&\\\paperiline
\papertableindex\papertablecval{8.44}\papertablecval{Orange}\\\paperiline
\papertableindex\papertablecval{9.18}\papertablecval{Green}\\\paperiline
\papertableindex\papertablecval{13.89}\papertablecval{Violet}\\\paperiline
\papertableindex\papertablecval{15.38}\papertablecval{Violet}\\\paperoline
\end{papertable}\vspace{-1.5em}}
{The scale readings and colours of the spectral lines of ``Unknown \#7''.}\vspace{1em}

\paperfig{UnknownNine}{\begin{papertable}{|[outer]I|[inner]C|[inner]C|[outer]}
\paperoline\papertableindexheader&\textsc{Scale}&\multirow{2}{*}{\textsc{Colour}}\\
\papertablecheaderunit{.02}{\hspace{-0.5ex}}&\\\paperiline
\papertableindex\papertablecval{TODO}\papertablecval{Red}\\\paperiline
\papertableindex\papertablecval{TODO}\papertablecval{Orange}\\\paperiline
\papertableindex\papertablecval{TODO}\papertablecval{Yellow}\\\paperiline
\papertableindex\papertablecval{TODO}\papertablecval{Green}\\\paperoline
\end{papertable}\vspace{-1.5em}}
{The scale readings and colours of the spectral lines of ``Unknown \#9''.}

\papersec{Analysis}

Analysis goes here

\papersec{Conclusion}

Conclusion goes here

\papersec{Sources}

\papersource{Serbanescu, R., Measuring atomic spectra, 2014}
\end{paper}
\end{document}