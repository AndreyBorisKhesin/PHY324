\input{header}
\physics
\begin{document}
\papertitle{Analysis of Atomic Spectra of Various Gases}
\paperauth{A}{Khesin}{Lab Technician}
\paperauth{P}{Zavyalova}{Data Analyst}
\paperdate{January 15, 2018}
\begin{paperabs}
In this experiment a spectrometer was used to study the spectra of various gases, determine the Rydberg constant, and identify several unknown substances.
The spectrometer was calibrated by performing a least-squares data fitting with the Hartmann relation as the model function. Upon calibrating the device, further measurements of the emission spectra of hydrogen and two unknown substances could be perfomed. 
The Rydberg constant was computed to be $(1.097\pm.003)\x\SI{E7}{\per\meter}$, with uncertainty arising from the reading of the vernier scale and the alignment of the crosshairs on the emission lines. 
The gases ``Unknown \#7'' and ``Unknown \#9'' were identified to be krypton and neon respectively. 
% a a a a a a a a a a a a a a a a a a a a a a a a a a a a a a a a a a a a a a a a a a a a a a a a a a a a a a a a a a a a a a a a a a a a a a a a a a a a a a a a a a a a a a a a a a a a a a a a a a a a a a a a a a a a a a a a a a a a a a a a a a a a a a a a a a a a a a a a a a a a a a a a a a a a a a a a a a a a a a a a a a a a a a a a a a a a a a a a a a a a a a a a a a a a a a a a a a a a a a a a
\end{paperabs}

\begin{paper}
\papersec{Introduction}

As an excited atom makes a transition to a lower state, a photon of discrete energy equal to the energy difference between the states is released. The collection of all possible transitions and corresponding emission lines makes up an atom's \textit{atomic spectrum}.

% a a a a a a a a a a a a a a a a a a a a a a a a a a a a a a a a a a a a a a a a a a a a a a a a a a a a a a a a a a a a a a a a a a a a a a a a a a a a a a a a a a a a a a a a a a a a a a a a a a a a a a a a a a a a a a a a a a a a a a a a a a a a a a a a a a a a a a a a a a a a a a a a a a a a

The purpose of this experiment was to use a prism spectrometer to identify
unknown gases based on their atomic spectra. %emission spectra. 
Furthermore, the Rydberg constant ($R_H$) was determined by examining the
spectral lines of hydrogen.
The apparatus consisted of a spectrometer, a sodium lamp, a tube of hydrogen
gas, a tube of helium gas, and two tubes marked ``Unknown
\#7'' and ``Unknown \#9''.

The spectrometer was calibrated by aligning the eyepiece and the crosshairs with the spectal lines of the sodium lamp and the Hartmann relation method was used to find the
relation between the observed wavelength and the scale reading on the
spectrometer.

By examining the spectral lines of hydrogen, the Rydberg constant was
determined with the following formula.

\papereq{Rydberg}{\frac1\lambda=R_H\pars{\frac14-\frac1{n^2}}}{Serbanescu}
\begin{paperwhere}
\papervar{\lambda}{observed wavelength}{\meter}
\papervar{R_H}{Rydberg constant}{\per\meter}
\papervar{n}{index of spectral line}{}
\end{paperwhere}

To determine the relation between the scale reading and the
observed wavelength, the Hartmann relation was used.\columnbreak

\papereq{Hartmann}{y=\frac{m}{\lambda-\lambda_0}+b}{Serbanescu}
\begin{paperwhere}
\papervar{y}{scale reading}{}
\papervar{m}{slope of fit}{\meter}
\papervar{\lambda_0}{fundamental wavelength of spectrometer}{\meter}
\papervar{b}{constant of fit}{}
\end{paperwhere}

For the spectrometer used in the experiment, the value of $\lambda_0$ was
provided as $(285.2\pm.4)\si{\nano\meter}$.

The experiment was setup as shown.
\paperfig{Setup}{\pdf{setup}}{The experimental setup.
The spectrometer (left) is pointing at a tube containing ``Unknown \#7'' (right).}
\papersec{Observations}

The vernier scale on the spectrometer had a reading error of
$\pm.02\si{\nano\meter}$.
The uncertainties in the wavelengths for the spectral lines of helium were
too small to consider.

The vernier scale readings were recorded for helium, hydrogen, ``Unknown \#7'',
and ``Unknown \#9''.
Additionally, the colours of the observed lines and matched wavelengths for the
spectral lines of helium were recorded.\pagebreak

\paperfig{Helium}{\begin{papertable}{|[outer]I|[inner]C|[inner]C|[inner]C|[outer]}
\paperoline\papertableindexheader&\textsc{Scale}&\textsc{Wavelength}&\multirow{2}{*}{\textsc{Colour}}\\
\papertablecheaderunit{.02}{\hspace{-0.5ex}}\papertablecheadersymbol{\lambda~(\si{\nano\meter})}&\\\paperiline
\papertableindex\papertablecval{6.66}\papertablecval{706.5}\papertablecval{Red}\\\paperiline
\papertableindex\papertablecval{7.06}\papertablecval{667.8}\papertablecval{Red}\\\paperiline
\papertableindex\papertablecval{8.31}\papertablecval{587.6}\papertablecval{Yellow}\\\paperiline
\papertableindex\papertablecval{11.09}\papertablecval{501.6}\papertablecval{Green}\\\paperiline
\papertableindex\papertablecval{11.50}\papertablecval{492.2}\papertablecval{Green}\\\paperiline
\papertableindex\papertablecval{12.58}\papertablecval{471.3}\papertablecval{Blue}\\\paperiline
\papertableindex\papertablecval{14.19}\papertablecval{447.1}\papertablecval{Blue}\\\paperoline
\end{papertable}\vspace{-1.5em}}
{The scale readings, matched wavelengths, and colours of the spectral lines of helium.}

\paperfig{Hydrogen}{\begin{papertable}{|[outer]I|[inner]C|[inner]C|[inner]C|[outer]}
\paperoline\papertableindexheader&\multirow{2}{*}{\textsc{Line Index}}&\textsc{Scale}&\multirow{2}{*}{\textsc{Colour}}\\
&\papertablecheaderunit{.02}{\hspace{-0.5ex}}&\\\paperiline
\papertableindex\papertablecval{3}\papertablecval{7.26}\papertablecval{Red}\\\paperiline
\papertableindex\papertablecval{4}\papertablecval{11.80}\papertablecval{Blue-Green}\\\paperiline
\papertableindex\papertablecval{5}\papertablecval{15.15}\papertablecval{Violet}\\\paperiline
\papertableindex\papertablecval{6}\papertablecval{17.75}\papertablecval{Violet}\\\paperoline
\end{papertable}\vspace{-1.5em}}
{The scale readings, indices, and colours of the spectral lines of hydrogen.}

\paperfig{UnknownSeven}{\begin{papertable}{|[outer]I|[inner]C|[inner]C|[outer]}
\paperoline\papertableindexheader&\textsc{Scale}&\multirow{2}{*}{\textsc{Colour}}\\
\papertablecheaderunit{.02}{\hspace{-0.5ex}}&\\\paperiline
\papertableindex\papertablecval{8.44}\papertablecval{Orange}\\\paperiline
\papertableindex\papertablecval{9.18}\papertablecval{Green}\\\paperiline
\papertableindex\papertablecval{13.89}\papertablecval{Violet}\\\paperiline
\papertableindex\papertablecval{15.38}\papertablecval{Violet}\\\paperoline
\end{papertable}\vspace{-1.5em}}
{The scale readings and colours of the spectral lines of ``Unknown \#7''.}

\paperfig{UnknownNine}{\begin{papertable}{|[outer]I|[inner]C|[inner]C|[outer]}
\paperoline\papertableindexheader&\textsc{Scale}&\multirow{2}{*}{\textsc{Colour}}\\
\papertablecheaderunit{.02}{\hspace{-0.5ex}}&\\\paperiline
\papertableindex\papertablecval{6.61}\papertablecval{Red}\\\paperiline
\papertableindex\papertablecval{8.05}\papertablecval{Orange}\\\paperiline
\papertableindex\papertablecval{8.50}\papertablecval{Yellow}\\\paperiline
\papertableindex\papertablecval{9.65}\papertablecval{Green}\\\paperoline
\end{papertable}\vspace{-1.5em}}
{The scale readings and colours of the spectral lines of ``Unknown \#9''.}

\papersec{Analysis}

The relationship between the observed wavelength and the
scale reading on the spectrometer was calculated.
This was done by subtracting $\lambda_0$ from all of the values in the wavelength column of
\figHelium and then taking the reciprocals to get the values of the independent
variables.
All uncertainties were propagated using the method of quadratures.
The scale column formed the values for the dependent variable.
Applying the fit yielded a slope of $(2.00\pm.02)\x\SI{E-6}{\meter}$ and a
y-intercept of $(1.83\pm.09)$.
The $R^2$ value for the fit is 0.999489.
The reduced $ \chi^2 $ obtained is $12.5828$, mostly due to two outliers, with residuals visualized below.

\paperfig{ScaleFit}{\begin{tikzpicture}[scale=0.95]
\begin{axis}[width=\linewidth,
tick label style={font=\fontsize{10pt}{1em}},
xlabel={\fontsize{10pt}{1em}\selectfont $\frac{1}{\lambda-\lambda_0}$ $(\SI{E6}{\per\meter})\rightarrow$},
ylabel={\fontsize{10pt}{1em}\selectfont $y\rightarrow$},
\paperaxis{Scale Reading vs. Modified Wavelength}{y}{x}{2}{5}{8}{15}{2}{2}]
\papergraphoutline{2}{5}{8}{15};
\addplot [color=black,only marks,mark=o,mark size=1.5,
error bars/.cd,
y dir=both,y explicit,
x dir=both,x explicit
] coordinates {
(2.374, 6.66) +- (0.002,0.02)
(2.614, 7.06) +- (0.003,0.02)
(3.307, 8.31) +- (0.004,0.02)
(4.621, 11.09) +- (0.009,0.02)
(4.831, 11.50) +- (0.009,0.02)
(5.37, 12.58) +- (0.01,0.02)
(6.18, 14.19) +- (0.02,0.02)
};
\addplot[no marks,
samples=250,
domain=2:8]
{2*x+1.83};
\end{axis}
\end{tikzpicture}\vspace{-1em}}
{The scale readings on the spectrometer plotted against the modified
wavelengths with the line of best fit.
The line has a function of $(2.00\pm.02)\x\SI{E-6}{}x+(1.83\pm.09)$ and an
\smash{$R^2$} value of 0.999489.
Error bars were included but were too small to be visible.}\\

\paperfig{Residuals}{\begin{tikzpicture}[scale = 0.95]
	\begin{axis}
		[width=\linewidth,
		tick label style={font=\fontsize{10pt}{1em}},
		xlabel={\fontsize{10pt}{1em}\selectfont $\lambda$ (\si{\nm}) $\rightarrow$},
		ylabel={\fontsize{10pt}{1em}\selectfont $ y - y_\text{fit} \rightarrow$},
		\paperaxis{Residuals vs. Wavelength}{y}{x}{450}{-0.15}{750}{0.15}{0}{2}]
		\papergraphoutline{450}{-0.15}{749}{0.15};
		\addplot[color = black, only marks, mark = o, mark size = 1.5, error bars/.cd,
		y dir = both, y explicit, x dir = both, x explicit] 
		coordinates {
			(706.5, 0.0836354) (667.8, 0.00358039) (587.6, -0.13242233)
			(501.6, 0.01987709) (492.2, 0.01029522) (471.3, 0.00550431)
			(447.1, 0.00952993)
		};
	\end{axis}
\end{tikzpicture}}{A plot of the residuals for the fitted data points. The reduced $ \chi^2 $ value for the fit is $12.5828$.}

Then, \eqHartmann was inverted to convert scale readings into wavelengths.

\papereq{Invert}{\lambda=\frac{m}{y-b}+\lambda_0}{}

By converting all the scale readings from \figHydrogen and plugging the
wavelengths and the line indices into \eqRydberg, four approximations
for the Rydberg constant were obtained.
They were averaged to get the final value for $R_H$, which was
$(1.097\pm.003)\x\SI{E7}{\per\meter}$, which was well within the range of the
experimental error from the accepted value of $\SI{1.09737316E7}{\per\meter}$.

The reversed formula was used to compute the wavelengths of the observed
spectral lines for the two unknown gases.
The following results were obtained for the two unknown gases.

\paperfig{UnknownSevenResult}{\begin{papertable}{|[outer]I|[inner]U|[inner]C|[outer]}
\paperoline\papertableindexheader&\multicolumn{2}{c|[inner]}{\textsc{Wavelength}}&\multirow{2}{*}{\textsc{Colour}}\\
\papertableuheaderunit{\si{\nano\meter}}&\\\paperiline
\papertableindex\papertableuval{\parbox{\widthof{TODO}}{\hfill588~~}}{\parbox{\widthof{TODO}}{~5}}\papertablecval{Orange}\\\paperiline
\papertableindex\papertableuval{\parbox{\widthof{TODO}}{\hfill557~~}}{\parbox{\widthof{TODO}}{~4}}\papertablecval{Green}\\\paperiline
\papertableindex\papertableuval{\parbox{\widthof{TODO}}{\hfill451~~}}{\parbox{\widthof{TODO}}{~2}}\papertablecval{Violet}\\\paperiline
\papertableindex\papertableuval{\parbox{\widthof{TODO}}{\hfill433~~}}{\parbox{\widthof{TODO}}{~2}}\papertablecval{Violet}\\\paperoline
\end{papertable}\vspace{-1.5em}}
{The wavelengths and colours of the spectral lines of ``Unknown \#7''.}\vspace{1em}

\paperfig{UnknownNineResult}{\begin{papertable}{|[outer]I|[inner]U|[inner]C|[outer]}
\paperoline\papertableindexheader&\multicolumn{2}{c|[inner]}{\textsc{Wavelength}}&\multirow{2}{*}{\textsc{Colour}}\\
\papertableuheaderunit{\si{\nano\meter}}&\\\paperiline
\papertableindex\papertableuval{\parbox{\widthof{TODO}}{\hfill704~~}}{\parbox{\widthof{TODO}}{~9}}\papertablecval{Red}\\\paperiline
\papertableindex\papertableuval{\parbox{\widthof{TODO}}{\hfill607~~}}{\parbox{\widthof{TODO}}{~6}}\papertablecval{Orange}\\\paperiline
\papertableindex\papertableuval{\parbox{\widthof{TODO}}{\hfill585~~}}{\parbox{\widthof{TODO}}{~5}}\papertablecval{Yellow}\\\paperiline
\papertableindex\papertableuval{\parbox{\widthof{TODO}}{\hfill541~~}}{\parbox{\widthof{TODO}}{~4}}\papertablecval{Green}\\\paperoline
\end{papertable}\vspace{-1.5em}}
{The wavelengths and colours of the spectral lines of ``Unknown \#9''.}\\

The colours of Unknowns 7 and 9 were recorded to be a silvery grey and a bright reddish-orange.
When this information was combined with the wavelengths of the brightest lines of the spectra for
various gases (Serbanescu), the identities of the gases could be determined.
The gas in the tube marked ``Unknown \#7'' was krypton and the gas in the tube marked ``Unknown \#9'' was neon.

For the last part of the experiment, the physical significance of $hcR_H$ was examined where $h$ is Planck's constant,
$c$ is the speed of light, and $R_H$ is the Rydberg constant.
The values of these constants were taken to be $h=\SI{6.62607004E-34}{\meter\squared\kilo\gram\per\second}$, $c=\SI{299792458}{\meter\per\second}$, and $R_H=\SI{1.09737316E7}{\per\meter}$.
This resulted in a value of $hcR_H=\SI{2.17987233E-18}{\joule}$.
Equivalently, this value was found to be equal to \SI{13.605693}{\electronvolt}.
This is precisely equal to the ionization energy of atomic hydrogen.

This phenomenon was explained as follows.
Energy obeys a well known relationship with wavelength.

\papereq{Energy}{E=\frac{hc}\lambda}{}
\begin{paperwhere}
\papervar{E}{energy}{\joule}
\papervar{h}{Planck's constant}{\meter\squared\kilo\gram\per\second}
\papervar{c}{speed of light}{\meter\per\second}
\end{paperwhere}

To have enough energy to escape an atom, an electron needs to move from an ``infinitely'' excited state to the ground state.
This was done by setting the value of $n$, the excited state to $\oo$ in the Lyman series.

\papereq{State}{\frac1\lambda=R_H\pars{1-\frac{1}{n^2}}}{Serbanescu}

By setting $n$ to $\oo$ we get that $\lambda$ is the reciprocal of $R_H$, so we get that $hcR_H=E$.
This means that $hcR_H$ precisely measures the amount of energy needed to move an electron from any excited state to the ground state, which is the ionization energy of hyrdrogen.

\papersec{Conclusion}

The experiment demonstrated successful use of the spectrometer.
First the spectrometer was calibrated using a sodium lamp and the scale reading function of the observed wavelength was derived from the spectral lines of helium.
Then the Rydberg constant was derived from spectral lines of hydrogen.

Two unknown gases were identified and the relationship between the Rydberg constant and the ionization energy of atomic hydrogen was established.
The main source of error in the experiment was the alignment of the crosshairs on the spectral lines and the reading of the vernier scale.

The derived function was found to be $(2.00\pm.02)\x\SI{E-6}{\meter}\frac{1}{\lambda-\lambda_0}+(1.83\pm.09)$.
The Rydberg constant was found to be $(1.097\pm.003)\x\SI{E7}{\per\meter}$.
The gases ``Unknown \#7'' and ``Unknown \#9'' were identified as krypton and neon, respectively.
Lastly, the value of $hcR_H$ was determined to be the ionization energy of atomic hydrogen.

\papersec{Sources}

\papersource{Serbanescu, R., Measuring atomic spectra, 2014}
\end{paper}
\end{document}