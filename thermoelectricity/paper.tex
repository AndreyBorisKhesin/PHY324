\input{header}

\physics
\begin{document}
\papertitle{Thermoelectricity}
\paperauth{A}{Khesin}{1002442029}
\paperauth{P}{Zavyalova}{1002345036}
\paperdate{February 12, 2018, Completed February 6, 2018}
\begin{paperabs}
	
	Abstract goes here
	
\end{paperabs}

\begin{paper}
	
\papersec{Introduction}
	
	A thermo-electric cooler (TEC) is a device that exploits Peltier and Seebeck effects to act as a solid-state heat pump. Composed of a set of metal-to-semiconductor thermocouple junctions, the TEC is capable of transferring heat from a cold environment to a warmer one, essentially acting as a reverse heat engine. 
	
	The Seebeck effect refers to the process of a temperature difference being converted into electricity. Given a circular loop composed of two different metals, it is evident that when maintained at the same temperature, the wires have different electron densities. As electrons move in the loop to correct the electron imbalance, net electric fields are created at the junctions. Further, introducing a temperature difference by keeping one junction at a constant temperature and increasing the temperature of the other gives electrons more kinetic energy and thus resulting in charge imbalances being corrected faster at the warmer junction. A net electric field is created in the loop; an electric potential difference and a current flow are established in the loop.
	
	The inverse of the Seebeck effect is referred to as the Peltier effect. As electric current passes through a loop made of two different junctions, a temperature difference between the two junctions is established. The junction where electron flow opposed the net electric field is being heated up as work is done by the electrons; at the other junction electron flow coincides with the direction of the net electric field and thus the junction is being cooled down. 
	
	On top of the above two effects underlying the functionality of a TEC, a variety of other processes are observed in the device. A significant example is ohmic heating, or the process of a circuit element with nonzero resistance being heated up as current flows through it. The process is simply described using the relationship below. 
	
	\papereq{OhmicHeating}{P = I ^ 2 R}{AF, 2014}
	\begin{paperwhere}
		\papervar{P}{heat production per unit time}{\watt}
		\papervar{I}{current}{\ampere}
		\papervar{R}{resistance}{\ohm}
	\end{paperwhere}

	In this experiment, the above physical principles were used to connect various aspects of TEC's performance. In steady-state operation of the TEC, the three elements to be examined were an ideal Peltier reversible heat pump, Ohmic resistive heating, and non-electronic heat flow from from the hot output port of the TEC to the cold input port.
	
	The experimental setup included the TEC, \textbf{four} multimeters, current balance, Variac transformer, filter capacitors to filter out noise produced by the TEC, and a \textbf{thermometer}, all arranged as shown below.
	
	\paperfig{Setup}{\pdf{setup}}{}

\paperfig{Circuit}
{\begin{tikzpicture}[scale=1.5]
\draw
(0,0) rectangle (2,1)
(1,0.5) node{\LARGE TEC}
(4.5,0.75) rectangle (5,1.25)
(4.75,1) node{CB}
(4.75,1.25) -- (4.75,1.5)
(4.5,1.5) rectangle (5,2)
(4.75,1.75) node{VT}
(0,1.5) rectangle (0.5,2)
(0.25,1.75) node{TS}
(0.17,1) -- (0.17,1.5)
(0.33,1) -- (0.33,1.5)
(0.25,1) node[below]{\small $T_1~T_2$}
(4.5,0.92) -| (4.25,0.33)
(4.5,1.08) -| (4.25,1.67)
(2,0.33) to[ammeter] (3.75,0.33) -- (4.25,0.33)
(2,0.33) node[left]{\small R}
(4.25,1.67) -- (3.75,1.67) to[voltmeter] (3.67,0.33)
(3.75,1.67) -| (2.5,0.67) -- (2,0.67) node[left]{\small R}
(1,1.5) rectangle (1.5,2)
(1.25,1.75) node{WO}
(1.25,1.5) -- (1.25,1)
(1.5,1.75) -- (4.5,1.75)
(0.25,0) -- (0.25,-1.5) to[ammeter] (1.75,-1.5) to[voltmeter] (3.75,-1.5) to[capacitor] (5,-1.5)
(0.25,0) node[above]{\small +}
(3.75,-1.5) -- (3.75,-0.5) to[capacitor] (5,-0.5) -- (5,-1.5)
(3.75,-0.5) -| (1.75,0) node[above]{\small \_}
(1.75,-1.5) -- (1.75,-1) -| (1,0) node[above]{\small +}
(0.25,-1.5) |- (5,-2) -- (5,-1.5)
(0.5,-1.5) node[above]{\small +}
(1.5,-1.5) node[above]{\small $-$}
(2.25,-1.5) node[above]{\small +}
(3.25,-1.5) node[above]{\small $-$}
(4,-0.5) node[above]{\small $-$}
(4.75,-0.5) node[above]{\small +}
(4,-1.5) node[above]{\small $-$}
(4.75,-1.5) node[above]{\small +};
\end{tikzpicture}}
{The circuit used in the experiment.
The lower ammeter and voltmeter are set to DC.
The upper ammeter and voltmeter are set to AC.
TEC is the thermo-electric cooler.
TS is the temperature sensor.
WO is the wall outlet.
VT is the Variac transformer.
CB is the current balance.
The filter capacitors are shown on the bottom right.
The points $T_1$ and $T_2$ are the temperature of the heat sink, $T_\text{out}$, and the temperature of the reservoir, $T_\text{in}$, respectively.
The two positive ports and the negative port of the TEC are indicated with pluses and a minus, respectively.
The resistor ports of the TEC are indicated with the letter R.
Positive and negative terminals were not shown on the upper ammeter and voltmeter as they do not affect the results by more than a sign.}
	
\papersec{Observations}
	
	To begin with, the temperatures of the input and output reservoirs and the ambient air were measured for several values of input power. To make the measurements, the terminals were open-circuited (obtained by unplugging one of the chords connecting the TEC to the DC ammeter). The ambient temperature was determined to be \( \pars{22.1 \pm 0.1} \si{\celsius} \), and was found to remain constant throughout the obtaining of the measurements. 
	
	\paperfig{TableOne}{\centering\begin{papertable}{|[outer]I|[inner]C|[inner]C|[inner]U|[inner]U|[outer]}\paperoline
			\papertableindexheader\papertablecheadersymbol{T_{out}}\papertablecheadersymbol{T_{in}}\papertableuheadersymbol{I_{in}}\papertableuheadersymbole{V_{in}}\\
			\papertablecheadersymbol{\pm.1\degree}\papertablecheadersymbol{\pm.1\degree}\papertableuheaderunit{\si{\ampere}}\papertableuheaderunite{\si{\volt}}\\\paperiline
			\papertableindex\papertablecval{23.0}\papertablecval{31.8}\papertableuval{1.54}{.01}\papertableuval{3.197}{.005}\\\paperiline
			\papertableindex\papertablecval{22.9}\papertablecval{31.4}\papertableuval{1.39}{.01}\papertableuval{2.884}{.004}\\\paperiline
			\papertableindex\papertablecval{22.9}\papertablecval{30.2}\papertableuval{1.20}{.01}\papertableuval{2.488}{.004}\\\paperiline
			\papertableindex\papertablecval{22.7}\papertablecval{27.4}\papertableuval{0.98}{.01}\papertableuval{2.044}{.004}\\\paperiline
			\papertableindex\papertablecval{22.5}\papertablecval{25.3}\papertableuval{0.693}{.008}\papertableuval{1.441}{.004}\\\paperiline
			\papertableindex\papertablecval{22.3}\papertablecval{23.3}\papertableuval{0.006}{.005}\papertableuval{0.014}{.003}\\\paperoline
	\end{papertable}\vspace{-0.5em}}{Measurements of $T_{out}$ ($\si{\celsius}$), $T_{in}$ ($\si{\celsius}$), $I_{in}$ ($\si{\ampere}$), and $V_{in}$ ($\si{\volt}$) for the first part of the experiment.}
	
	Further, after allowing the system to cool off, the TEC was operated with an approximately fixed current supplied from the built-in power supply. The Variac transformer was turned off; measurements of voltage across the TEC and the current supplied to it, as well as the temperatures of the output and the input reservoir were taken.

	\bgroup\centering\begin{papertable}{|[outer]I|[inner]C|[inner]C|[inner]U|[inner]U|[outer]}\paperoline
	\papertableindexheader\papertablecheadersymbol{T_{out}}\papertablecheadersymbol{T_{in}}\papertableuheadersymbol{I_d}\papertableuheadersymbole{V_d}\\
	\papertablecheadersymbol{\pm.1\degree}\papertablecheadersymbol{\pm.1\degree}\papertableuheaderunit{\si{\ampere}}\papertableuheaderunite{\si{\volt}}\\\paperiline
\papertableindex\papertablecval{22.2}\papertablecval{20.6}\papertableuval{2.68}{.07}\papertableuval{9.77}{.03}\\\paperiline
\papertableindex\papertablecval{24.7}\papertablecval{16.7}\papertableuval{2.72}{.07}\papertableuval{9.72}{.03}\\\paperiline
\papertableindex\papertablecval{25.7}\papertablecval{15.0}\papertableuval{2.70}{.07}\papertableuval{9.47}{.02}\\\paperiline
\papertableindex\papertablecval{26.4}\papertablecval{13.4}\papertableuval{2.69}{.07}\papertableuval{9.49}{.02}\\\paperiline
\papertableindex\papertablecval{27.1}\papertablecval{11.3}\papertableuval{2.65}{.07}\papertableuval{9.29}{.02}\\\paperiline
\papertableindex\papertablecval{27.3}\papertablecval{10.4}\papertableuval{2.63}{.07}\papertableuval{9.21}{.02}\\\paperiline
\papertableindex\papertablecval{27.6}\papertablecval{9.1}\papertableuval{2.62}{.07}\papertableuval{9.16}{.02}\\\paperiline
\papertableindex\papertablecval{27.8}\papertablecval{8.0}\papertableuval{2.65}{.07}\papertableuval{9.22}{.02}\\\paperiline
\papertableindex\papertablecval{28.0}\papertablecval{6.8}\papertableuval{2.59}{.07}\papertableuval{9.00}{.02}\\\paperiline
\papertableindex\papertablecval{28.1}\papertablecval{5.9}\papertableuval{2.75}{.08}\papertableuval{9.57}{.02}\\\paperiline
\papertableindex\papertablecval{28.2}\papertablecval{4.8}\papertableuval{2.78}{.08}\papertableuval{9.65}{.03}\\\paperiline
\papertableindex\papertablecval{28.4}\papertablecval{4.0}\papertableuval{2.78}{.08}\papertableuval{9.62}{.03}\\\paperiline
\papertableindex\papertablecval{28.4}\papertablecval{3.1}\papertableuval{2.79}{.08}\papertableuval{9.65}{.03}\\\paperiline
\papertableindex\papertablecval{28.5}\papertablecval{2.3}\papertableuval{2.79}{.08}\papertableuval{9.66}{.03}\\\paperiline
\papertableindex\papertablecval{28.5}\papertablecval{1.5}\papertableuval{2.80}{.08}\papertableuval{9.68}{.03}\\\paperiline
\papertableindex\papertablecval{28.6}\papertablecval{0.7}\papertableuval{2.80}{.08}\papertableuval{9.69}{.03}\\\paperiline
\papertableindex\papertablecval{28.6}\papertablecval{0.1}\papertableuval{2.80}{.08}\papertableuval{9.67}{.03}\\\paperiline
\papertableindex\papertablecval{28.6}\papertablecval{-0.4}\papertableuval{2.80}{.08}\papertableuval{9.67}{.03}\\\paperiline
\papertableindex\papertablecval{28.6}\papertablecval{-1.1}\papertableuval{2.81}{.08}\papertableuval{9.71}{.03}\\\paperiline
\papertableindex\papertablecval{28.6}\papertablecval{-1.7}\papertableuval{2.81}{.08}\papertableuval{9.71}{.03}\\\paperiline
\papertableindex\papertablecval{28.6}\papertablecval{-2.7}\papertableuval{2.81}{.08}\papertableuval{9.69}{.03}\\\paperiline
\papertableindex\papertablecval{28.6}\papertablecval{-3.3}\papertableuval{2.82}{.08}\papertableuval{9.73}{.03}\\\paperiline
\papertableindex\papertablecval{28.6}\papertablecval{-3.7}\papertableuval{2.81}{.08}\papertableuval{9.68}{.03}\\\paperiline
\papertableindex\papertablecval{28.6}\papertablecval{-4.2}\papertableuval{2.80}{.08}\papertableuval{9.67}{.03}\\\paperiline
\papertableindex\papertablecval{28.5}\papertablecval{-4.7}\papertableuval{2.74}{.07}\papertableuval{9.41}{.02}\\\paperiline
\papertableindex\papertablecval{28.3}\papertablecval{-5.2}\papertableuval{2.73}{.07}\papertableuval{9.34}{.02}\\\paperiline
\papertableindex\papertablecval{28.2}\papertablecval{-5.5}\papertableuval{2.70}{.07}\papertableuval{9.28}{.02}\\\paperiline
\papertableindex\papertablecval{28.2}\papertablecval{-5.7}\papertableuval{2.70}{.07}\papertableuval{9.29}{.02}\\\paperoline\end{papertable}\egroup

\paperfig{PartTwoA}{\centering\begin{papertable}{|[outer]I|[inner]C|[inner]C|[inner]U|[inner]U|[outer]}\paperoline
	\papertableindexheader\papertablecheadersymbol{T_{out}}\papertablecheadersymbol{T_{in}}\papertableuheadersymbol{I_d}\papertableuheadersymbole{V_d}\\
	\papertablecheadersymbol{\pm.1\degree}\papertablecheadersymbol{\pm.1\degree}\papertableuheaderunit{\si{\ampere}}\papertableuheaderunite{\si{\volt}}\\\paperiline
\stepcounter{papertableindexcounter}\stepcounter{papertableindexcounter}\stepcounter{papertableindexcounter}\stepcounter{papertableindexcounter}\stepcounter{papertableindexcounter}\stepcounter{papertableindexcounter}\stepcounter{papertableindexcounter}\stepcounter{papertableindexcounter}\stepcounter{papertableindexcounter}\stepcounter{papertableindexcounter}\stepcounter{papertableindexcounter}\stepcounter{papertableindexcounter}\stepcounter{papertableindexcounter}\stepcounter{papertableindexcounter}\stepcounter{papertableindexcounter}\stepcounter{papertableindexcounter}\stepcounter{papertableindexcounter}\stepcounter{papertableindexcounter}\stepcounter{papertableindexcounter}\stepcounter{papertableindexcounter}\stepcounter{papertableindexcounter}\stepcounter{papertableindexcounter}\stepcounter{papertableindexcounter}\stepcounter{papertableindexcounter}\stepcounter{papertableindexcounter}\stepcounter{papertableindexcounter}\stepcounter{papertableindexcounter}\stepcounter{papertableindexcounter}\papertableindex\papertablecval{28.1}\papertablecval{-5.9}\papertableuval{2.71}{.07}\papertableuval{9.33}{.02}\\\paperiline
\papertableindex\papertablecval{28.1}\papertablecval{-6.2}\papertableuval{2.63}{.07}\papertableuval{9.21}{.02}\\\paperiline
\papertableindex\papertablecval{28.1}\papertablecval{-6.3}\papertableuval{2.67}{.07}\papertableuval{9.21}{.02}\\\paperiline
\papertableindex\papertablecval{28.0}\papertablecval{-6.5}\papertableuval{2.68}{.07}\papertableuval{9.20}{.02}\\\paperiline
\papertableindex\papertablecval{28.0}\papertablecval{-6.7}\papertableuval{2.71}{.07}\papertableuval{9.34}{.02}\\\paperiline
\papertableindex\papertablecval{27.9}\papertablecval{-6.9}\papertableuval{2.80}{.08}\papertableuval{9.72}{.03}\\\paperiline
\papertableindex\papertablecval{27.9}\papertablecval{-7.1}\papertableuval{2.66}{.07}\papertableuval{9.21}{.02}\\\paperiline
\papertableindex\papertablecval{27.9}\papertablecval{-7.2}\papertableuval{2.65}{.07}\papertableuval{9.15}{.02}\\\paperiline
\papertableindex\papertablecval{27.9}\papertablecval{-7.4}\papertableuval{2.63}{.07}\papertableuval{9.14}{.02}\\\paperiline
\papertableindex\papertablecval{27.8}\papertablecval{-7.4}\papertableuval{2.85}{.08}\papertableuval{9.98}{.03}\\\paperiline
\papertableindex\papertablecval{27.9}\papertablecval{-7.6}\papertableuval{2.86}{.08}\papertableuval{10.05}{.03}\\\paperiline
\papertableindex\papertablecval{28.1}\papertablecval{-7.9}\papertableuval{2.86}{.08}\papertableuval{10.05}{.03}\\\paperiline
\papertableindex\papertablecval{28.1}\papertablecval{-8.1}\papertableuval{2.87}{.08}\papertableuval{10.07}{.03}\\\paperiline
\papertableindex\papertablecval{28.2}\papertablecval{-8.2}\papertableuval{2.84}{.08}\papertableuval{9.99}{.03}\\\paperiline
\papertableindex\papertablecval{28.3}\papertablecval{-8.3}\papertableuval{2.87}{.08}\papertableuval{10.13}{.03}\\\paperiline
\papertableindex\papertablecval{28.3}\papertablecval{-8.5}\papertableuval{2.87}{.08}\papertableuval{10.14}{.03}\\\paperiline
\papertableindex\papertablecval{28.3}\papertablecval{-8.6}\papertableuval{2.87}{.08}\papertableuval{10.15}{.03}\\\paperiline
\papertableindex\papertablecval{28.3}\papertablecval{-8.7}\papertableuval{2.88}{.08}\papertableuval{10.21}{.03}\\\paperiline
\papertableindex\papertablecval{28.3}\papertablecval{-8.7}\papertableuval{2.87}{.08}\papertableuval{10.21}{.03}\\\paperiline
\papertableindex\papertablecval{28.3}\papertablecval{-8.6}\papertableuval{2.84}{.08}\papertableuval{10.09}{.03}\\\paperiline
\papertableindex\papertablecval{28.3}\papertablecval{-8.8}\papertableuval{2.72}{.07}\papertableuval{9.69}{.03}\\\paperiline
\papertableindex\papertablecval{28.3}\papertablecval{-8.9}\papertableuval{2.75}{.08}\papertableuval{9.79}{.03}\\\paperiline
\papertableindex\papertablecval{28.2}\papertablecval{-8.9}\papertableuval{2.71}{.07}\papertableuval{9.66}{.03}\\\paperiline
\papertableindex\papertablecval{28.1}\papertablecval{-8.9}\papertableuval{2.83}{.08}\papertableuval{10.17}{.03}\\\paperiline
\papertableindex\papertablecval{28.1}\papertablecval{-8.9}\papertableuval{2.83}{.08}\papertableuval{10.19}{.03}\\\paperiline
\papertableindex\papertablecval{28.1}\papertablecval{-9.0}\papertableuval{2.84}{.08}\papertableuval{10.25}{.03}\\\paperiline
\papertableindex\papertablecval{28.1}\papertablecval{-9.0}\papertableuval{2.84}{.08}\papertableuval{10.27}{.03}\\\paperiline
\papertableindex\papertablecval{28.2}\papertablecval{-9.1}\papertableuval{2.85}{.08}\papertableuval{10.31}{.03}\\\paperoline
\end{papertable}\vspace{-0.5em}}{Measurements of $T_{out}$ ($\si{\celsius}$), $T_d$ ($\si{\celsius}$), $I_d$ ($\si{\ampere}$), and $V_{in}$ ($\si{\volt}$) for the first part of the second part of the experiment.}

	Finally, after reaching a \( 30 \si{\celsius} \) temperature difference, the Variac transformer was turned on to supply various power inputs to the system. Allowing the system to reach a semi-stable state (waiting roughly two minutes after changing the power), reservoir temperatures, input and output power, and the Seebeck voltage were measured. The Seebeck voltage was obtained by turning off the TEC and recording the highest reading of continuously running-off potential. 

\papersec{Analysis} 
	
	To visualize the data with the open-circuited terminals, temperature difference was plotted against the input power. Uncertainties were propagated using the method of quadratures.
	
	\paperfig{part1}{\pdf{1}}{Temperature difference plotted against the power for the open-circuited terminals. The error bars were included but were too small to be visible.}
	
	To visualize the results for the fixed current setup supplied by the built-in power supply, temperature difference and the electrical power input were plotted against time. All uncertainties were propagates using the method of quadratures.
	
	\paperfig{part2a}{\pdf{part2a-data}}{Power and temperature difference plotted against time. Uncertainties were included but were too small to be visible.}
	
	To visualize the results for various power inputs supplied by the Variac transformer, reservoir temperatures, power, and the Seebeck voltage were plotted versus the input power. All uncertainties were propagates using the method of quadratures.
	
	\paperfig{part2b}{\pdf{part2b-data}}{Temperature vs. input power (left), total electrical power vs. input power (middle) and total voltage across the TEC vs. input power (right). Uncertainties were included but were too small to be visible.}
	
	
	
\papersec{Conclusion}

	Conclusion goes here
	
\papersec{Sources}

	\papersource{Albanelli, P., Fomichev, S., Thermoelectricity, 2014}

\end{paper}
\end{document}