\input{header}

\physics
\begin{document}
\papertitle{Thermoelectricity}
\paperauth{A}{Khesin}{1002442029}
\paperauth{P}{Zavyalova}{1002345036}
\paperdate{February 12, 2018, Completed February 6, 2018}
\begin{paperabs}
	
	Abstract goes here
	
\end{paperabs}

\begin{paper}
	
\papersec{Introduction}
	
	Introduction goes here 

\paperfig{Circuit}
{\begin{tikzpicture}[scale=1.5]
\draw
(0,0) rectangle (2,1)
(1,0.5) node{\LARGE TEC}
(4.5,0.75) rectangle (5,1.25)
(4.75,1) node{CB}
(4.75,1.25) -- (4.75,1.5)
(4.5,1.5) rectangle (5,2)
(4.75,1.75) node{VT}
(0,1.5) rectangle (0.5,2)
(0.25,1.75) node{TS}
(0.17,1) -- (0.17,1.5)
(0.33,1) -- (0.33,1.5)
(0.25,1) node[below]{\small $T_1~T_2$}
(4.5,0.92) -| (4.25,0.33)
(4.5,1.08) -| (4.25,1.67)
(2,0.33) to[ammeter] (3.75,0.33) -- (4.25,0.33)
(2,0.33) node[left]{\small R}
(4.25,1.67) -- (3.75,1.67) to[voltmeter] (3.67,0.33)
(3.75,1.67) -| (2.5,0.67) -- (2,0.67) node[left]{\small R}
(1,1.5) rectangle (1.5,2)
(1.25,1.75) node{WO}
(1.25,1.5) -- (1.25,1)
(1.5,1.75) -- (4.5,1.75)
(0.25,0) -- (0.25,-1.5) to[ammeter] (1.75,-1.5) to[voltmeter] (3.75,-1.5) to[capacitor] (5,-1.5)
(0.25,0) node[above]{\small +}
(3.75,-1.5) -- (3.75,-0.5) to[capacitor] (5,-0.5) -- (5,-1.5)
(3.75,-0.5) -| (1.75,0) node[above]{\small \_}
(1.75,-1.5) -- (1.75,-1) -| (1,0) node[above]{\small +}
(0.25,-1.5) |- (5,-2) -- (5,-1.5)
(0.5,-1.5) node[above]{\small +}
(1.5,-1.5) node[above]{\small $-$}
(2.25,-1.5) node[above]{\small +}
(3.25,-1.5) node[above]{\small $-$}
(4,-0.5) node[above]{\small $-$}
(4.75,-0.5) node[above]{\small +}
(4,-1.5) node[above]{\small $-$}
(4.75,-1.5) node[above]{\small +};
\end{tikzpicture}}
{The circuit used in the experiment.
The lower ammeter and voltmeter are set to DC.
The upper ammeter and voltmeter are set to AC.
TEC is the thermo-electric cooler.
TS is the temperature sensor.
WO is the wall outlet.
VT is the Variac transformer.
CB is the current balance.
The filter capacitors are shown on the bottom right.
The points $T_1$ and $T_2$ are the temperature of the heat sink, $T_\text{out}$, and the temperature of the reservoir, $T_\text{in}$, respectively.
The two positive ports and the negative port of the TEC are indicated with pluses and a minus, respectively.
The resistor ports of the TEC are indicated with the letter R.
Positive and negative terminals were not shown on the upper ammeter and voltmeter as they do not affect the results by more than a sign.}
	
\papersec{Observations}
	
	Observations go here
	
\papersec{Analysis} 
	
	Analysis goes here
	
\papersec{Conclusion}

	Conclusion goes here
	
\papersec{Sources}

	\papersource{Albanelli, P., Fomichev, S., Thermoelectricity, 2014}

\end{paper}
\end{document}