\input{header}
\physics
\begin{document}
\papertitle{The Thermocouple}
\paperauth{A}{Khesin}{1002442029}
\paperauth{N}{Rahnamaei}{1002552686}
\paperauth{P}{Zavyalova}{1002345036}
\paperdate{January 24, 2018}
\begin{paperabs}
	
	Abstract goes here
	
\end{paperabs}

\begin{paper}
	
\papersec{Introduction}

	Thermocouples convert a temperature difference into an electromotive force (EMF), and are made of two wires of dissimilar metals which are joined at each end. The wires in a temperature gradient will allow the free carriers at the hot end have more kinetic energy and tend to diffuse towards the cold end, which will in turn create an electric field which has a averse tendency the heat flow. In this experiment, the temperature difference was established between an ice bath staying at constant temperature, and another body of water with varying temperature. 
	
	
	
	The circuit used in the experiment was as follows.
	
	\paperfig{Circuit}{\center
	\begin{tikzpicture}[scale=0.7,transform shape]
	\draw (1,0) to[voltmeter] (3,0) -- (3,1) -- (4,1) -- (4,2) -- (5,2)
	(4,2) -- (4,3) -- (0,3) -- (0,2) -- (-1,2) -- (0,2) -- (0,1) -- (1,1) -- (1,0);
	\draw (0.3,1.5) node[right]{$B$};
	\draw (3.3,1.5) node[right]{$B$};
	\draw (1.8,2.5) node[right]{$A$};
	\draw (-1,2) node[left]{$T_1$};
	\draw (5,2) node[right]{$T_2$};
	\end{tikzpicture}
	}{The circuit diagram used in the experiment.
	The wire marked $A$ was made of Constantan and the wires marked $B$ were made of Copper.
	The temperature difference was being measured between points $T_1$ and $T_2$.
	The voltmeter shown at the bottom was used to measure the potential differences in the circuit.}	
	
\papersec{Observations}
	
	The temperature $T_1$ in the ice water bath was kept at a constant $(0.0\pm.5)\degree C$.
	This value was occasionally remeasured during the experiment to make sure that it was unchanged.
	The recorded values for the temperatures at $T_2$ and for the potential differences $V$ were recorded in the table below.
	
	\paperfig{Table}{\begin{papertable}{|[outer]I|[inner]C|[inner]C|[outer]}\paperoline
	\papertableindexheader&\multirow{2}{*}{\textsc{Temperature}\vspace{0.5em}}&\multirow{2}{*}{\textsc{Potential}\vspace{0.5em}}\\
	\papertablecheadersymbol{T_2$ ($\si{\celsius}$) $(\pm.5\si{\celsius})}\papertablecheadersymbol{V$ ($\si{\volt}$) $(\pm.1\si{\volt})}\\\paperiline
	\papertableindex\papertablecval{97.0}\papertablecval{4.0}\\\paperiline
	\papertableindex\papertablecval{79.5}\papertablecval{3.3}\\\paperiline
	\papertableindex\papertablecval{70.0}\papertablecval{2.9}\\\paperiline
	\papertableindex\papertablecval{60.0}\papertablecval{2.4}\\\paperiline
	\papertableindex\papertablecval{46.5}\papertablecval{1.9}\\\paperiline
	\papertableindex\papertablecval{38.5}\papertablecval{1.5}\\\paperiline
	\papertableindex\papertablecval{30.0}\papertablecval{1.2}\\\paperiline
	\papertableindex\papertablecval{24.5}\papertablecval{1.0}\\\paperiline
	\papertableindex\papertablecval{17.0}\papertablecval{0.7}\\\paperiline
	\papertableindex\papertablecval{9.5}\papertablecval{0.3}\\\paperiline
	\papertableindex\papertablecval{3.0}\papertablecval{0.1}\\\paperiline
	\papertableindex\papertablecval{0.0}\papertablecval{0.0}\\\paperoline
	\end{papertable}\vspace{-1em}}{The observations recorded during the experiment of the relationship between $T_2$ and $V$.
	The value of $T_1$ was not included in the table as it was constant.}
		
\papersec{Analysis} 
	
	To calibrate the thermocouple, measured EMF values were plotted against the temperature difference between the junctions. Upon fitting the data using least squares method, the Seebeck coefficient was determined to be \( \pars{0.0410 \pm 0.0003} \si{\milli\volt\per\kelvin} \). The value of reduced \( \chi ^2 \) for this fit was \( 0.202632 \). The calibration curve, fit, and residuals are visualized below. 
	
	\paperfig{CalibrationCurve}{\pdf{ThermocoupleCalibration}}{EMF and residuals plotted against the temperature difference between the junctions. The curve has the function \(  \pars{0.0410 \pm 0.0003} \Delta T \) and an \( R^2 \) value of \( 0.998821 \).}
	
	In this experiment, the output resistance of the thermocouple was measured.
	The value of this resistance was calculated to be $(1.8\pm.4)\si{\ohm}$.

	The precision of any measurements made in this experiment could have been increased by using several thermocouples in series.
	Since the differences in temperature between the endpoints of the thermocouples would have added up to the difference between the first and last endpoint, this would have measured the same value of temperature difference, but the fact that the temperature would be measured in many smaller pieces would increase the precision of the measurement.
	The situation would be identical if a large distance more precisely by measuring many smaller intervals which together add to the longer distance.
	Thus, there would be a advantage to connecting several thermocouples in series.
	
\papersec{Conclusion}

	Conclusion goes here

\papersec{Sources}

	\papersource{Serbanescu, R., The Thermocouple, 2013}

\end{paper}

\end{document}