\input{header}
\physics
\begin{document}
\papertitle{The Thermocouple}
\paperauth{A}{Khesin}{1002442029}
\paperauth{N}{Rahnamaei}{STUDENT \#}
\paperauth{P}{Zavyalova}{1002345036}
\paperdate{January 24, 2018}
\begin{paperabs}
	
	Abstract goes here
	
\end{paperabs}

\begin{paper}
	
\papersec{Introduction}

	Thermocouples convert a temperature difference into an electromotive force (EMF), and are made of two wires of dissimilar metals which are joined at each end. The wires in a temperature gradient will allow the free carriers at the hot end have more kinetic energy and tend to diffuse towards the cold end, which will in turn create an electric field which has a averse tendency the heat flow. In this experiment, the temperature difference is measure of an ice bath which stays at constant temperature, and another body of water with temperature which will vary. 
	
\papersec{Observations}

<<<<<<< HEAD
	Observations
=======
	Observations go here
	
	The circuit used in the experiment was as follows.
	
	\paperfig{Circuit}{\center
	\begin{tikzpicture}[scale=0.7,transform shape]
	\draw (1,0) to[voltmeter] (3,0) -- (3,1) -- (4,1) -- (4,2) -- (5,2)
	(4,2) -- (4,3) -- (0,3) -- (0,2) -- (-1,2) -- (0,2) -- (0,1) -- (1,1) -- (1,0);
	\draw (0.3,1.5) node[right]{$B$};
	\draw (3.3,1.5) node[right]{$B$};
	\draw (1.8,2.5) node[right]{$A$};
	\draw (-1,2) node[left]{$T_1$};
	\draw (5,2) node[right]{$T_2$};
	\end{tikzpicture}
	}{The circuit diagram used in the experiment.
	The wire marked $A$ was made of Constantan and the wires marked $B$ were made of Copper.
	The temperature difference was being measured between points $T_1$ and $T_2$.}
>>>>>>> 074b855c541928de6037c77a07bd2a79f72ad005
		
\papersec{Analysis} 
	
	To calibrate the thermocouple, measured EMF values were plotted against the temperature difference between the junctions. Upon fitting the data using least squares method, the Seebeck coefficient was determined to be \( \pars{0.0410 \pm 0.0003} \si{\milli\volt\per\kelvin} \). The value of reduced \( \chi ^2 \) for this fit was \( 0.202632 \). The calibration curve, fit, and residuals are visualized below. 
	
	\paperfig{CalibrationCurve}{\pdf{ThermocoupleCalibration}}{EMF and residuals plotted against the temperature difference between the junctions. The curve has the function \(  \pars{0.0410 \pm 0.0003} \Delta T \) and an \( R^2 \) value of \( 0.998821 \).}
	
\papersec{Conclusion}

	Conclusion goes here

\papersec{Sources}

	\papersource{Serbanescu, R., The Thermocouple, 2013}

\end{paper}

\end{document}