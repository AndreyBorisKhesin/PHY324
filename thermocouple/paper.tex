\input{header}
\physics
\begin{document}
\papertitle{The Thermocouple}
\paperauth{A}{Khesin}{1002442029}
\paperauth{N}{Rahnamaei}{STUDENT \#}
\paperauth{P}{Zavyalova}{1002345036}
\paperdate{January 24, 2018}
\begin{paperabs}
	
	Abstract goes here
	
\end{paperabs}

\begin{paper}
	
\papersec{Introduction}

	Introduction goes here
	
\papersec{Observations}

	Observations go here
	
	The circuit used in the experiment was as follows.
	
	\paperfig{Circuit}{\center
	\begin{tikzpicture}[scale=0.7,transform shape]
	\draw (1,0) to[voltmeter] (3,0) -- (3,1) -- (4,1) -- (4,2) -- (5,2)
	(4,2) -- (4,3) -- (0,3) -- (0,2) -- (-1,2) -- (0,2) -- (0,1) -- (1,1) -- (1,0);
	\draw (0.3,1.5) node[right]{$B$};
	\draw (3.3,1.5) node[right]{$B$};
	\draw (1.8,2.5) node[right]{$A$};
	\draw (-1,2) node[left]{$T_1$};
	\draw (5,2) node[right]{$T_2$};
	\end{tikzpicture}
	}{The circuit diagram used in the experiment.
	The wire marked $A$ was made of Constantan and the wires marked $B$ were made of Copper.
	The temperature difference was being measured between points $T_1$ and $T_2$.
	The voltmeter shown at the bottom was used to measure the potential differences in the circuit.}
	
	The temperature $T_1$ in the ice water bath was kept at a constant $(0.0\pm.5)\degree C$.
	This value was occasionally remeasured during the experiment to make sure that it was unchanged.
	The recorded values for the temperatures at $T_2$ and for the potential differences $V$ were recorded in the table below.
	
	\paperfig{Table}{\begin{papertable}{|[outer]I|[inner]C|[inner]C|[outer]}\paperoline
	\papertableindexheader&\multirow{2}{*}{\textsc{Temperature}\vspace{0.5em}}&\multirow{2}{*}{\textsc{Potential}\vspace{0.5em}}\\
	\papertablecheadersymbol{T_2$ ($\si{\kelvin}$) $(\pm.5\si{\kelvin})}\papertablecheadersymbol{V$ ($\si{\volt}$) $(\pm.1\si{\volt})}\\\paperiline
	\papertableindex\papertablecval{97.0}\papertablecval{4.0}\\\paperiline
	\papertableindex\papertablecval{79.5}\papertablecval{3.3}\\\paperiline
	\papertableindex\papertablecval{70.0}\papertablecval{2.9}\\\paperiline
	\papertableindex\papertablecval{60.0}\papertablecval{2.4}\\\paperiline
	\papertableindex\papertablecval{46.5}\papertablecval{1.9}\\\paperiline
	\papertableindex\papertablecval{38.5}\papertablecval{1.5}\\\paperiline
	\papertableindex\papertablecval{30.0}\papertablecval{1.2}\\\paperiline
	\papertableindex\papertablecval{24.5}\papertablecval{1.0}\\\paperiline
	\papertableindex\papertablecval{17.0}\papertablecval{0.7}\\\paperiline
	\papertableindex\papertablecval{9.5}\papertablecval{0.3}\\\paperiline
	\papertableindex\papertablecval{3.0}\papertablecval{0.1}\\\paperiline
	\papertableindex\papertablecval{0.0}\papertablecval{0.0}\\\paperoline
	\end{papertable}\vspace{-1em}}{The observations recorded during the experiment of the relationship between $T_2$ and $V$.
	The value of $T_1$ was not included in the table as it was constant.}
		
\papersec{Analysis} 
	
	To calibrate the thermocouple, measured EMF values were plotted against the temperature difference between the junctions. Upon fitting the data using least squares method, the Seebeck coefficient was determined to be \( \pars{0.0410 \pm 0.0003} \si{\milli\volt\per\kelvin} \). The value of reduced \( \chi ^2 \) for this fit was \( 0.202632 \). The calibration curve, fit, and residuals are visualized below. 
	
	\paperfig{CalibrationCurve}{\pdf{ThermocoupleCalibration}}{EMF and residuals plotted against the temperature difference between the junctions. The curve has the function \(  \pars{0.0410 \pm 0.0003} \Delta T \) and an \( R^2 \) value of \( 0.998821 \).}
	
\papersec{Conclusion}

	Conclusion goes here

\papersec{Sources}

	\papersource{Serbanescu, R., The Thermocouple, 2013}

\end{paper}

\end{document}