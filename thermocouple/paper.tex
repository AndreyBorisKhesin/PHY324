\input{header}
\physics
\begin{document}
\papertitle{The Thermocouple}
\paperauth{A}{Khesin}{1002442029}
\paperauth{N}{Rahnamaei}{STUDENT \#}
\paperauth{P}{Zavyalova}{1002345036}
\paperdate{January 24, 2018}
\begin{paperabs}
	
	Abstract goes here
	
\end{paperabs}

\begin{paper}
	
\papersec{Introduction}

	Introduction goes here
	
\papersec{Observations}

	Observations go here
		
\papersec{Analysis} 
	
	To calibrate the thermocouple, measured EMF values were plotted against the temperature difference between the junctions. Upon fitting the data using least squares method, the Seebeck coefficient was determined to be \( \pars{0.0410 \pm 0.0003} \si{\milli\volt\per\kelvin} \). The value of reduced \( \chi ^2 \) for this fit was \( 0.202632 \). The calibration curve, fit, and residuals are visualized below. 
	
	\paperfig{CalibrationCurve}{\pdf{ThermocoupleCalibration}}{EMF and residuals plotted against the temperature difference between the junctions. The curve has the function \(  \pars{0.0410 \pm 0.0003} \Delta T \) and an \( R^2 \) value of \( 0.998821 \).}
	
\papersec{Conclusion}

	Conclusion goes here

\papersec{Sources}

	\papersource{Serbanescu, R., The Thermocouple, 2013}

\end{paper}

\end{document}