\input{header}
\physics
\begin{document}
\papertitle{Non-linear Circuit Elements}
\paperauth{A}{Khesin}{}
\paperauth{P}{Zavyalova}{}
\paperdate{January 17, 2018}
\begin{paperabs}
	
	Abstract goes here
	
\end{paperabs}

\begin{paper}
	
\papersec{Introduction}

	The purpose of this experiment was to study the relationship between voltage and current for a thermistor and an incandescent lamp. Both of these circuit elements are nonlinear; for a thermistor, the resistance is a function of temperature and obeys the relation
	
	\papereq{Thermistor}{R = R_0e^{ -\frac{T}{T_0} }}{Serbanescu}
	\begin{paperwhere}
		\papervar{T}{absolute temperature}{\kelvin}
		\papervar{T_0}{constant of device}{\kelvin}
		\papervar{R_0}{constant of device}{\ohm}
	\end{paperwhere}

	The relationship was investigated using a multimeter, a thermistor, a thermometer and a thermal flask for maintaining the thermistor at different temperatures, set up as shown:
	
	\paperfig{Setup}{\pdf{setup1}}{The experimental setup (thermistor).}
	
	The temperature of the thermistor was decreased by adding cold water to the thermal flask. 
	
	The tungsten filament in a light bulb follows Ohm's law, 
	
	\papereq{Ohm}{ R = \frac{V}{I} }{Serbanescu}
	\begin{paperwhere}
		\papervar{R}{resistance}{\ohm}
		\papervar{V}{potential}{\volt}
		\papervar{I}{current}{\ampere}
	\end{paperwhere}
	
	However, as the temperature increases, the resistance of the metal increases and above a certain temperature becomes proportional to the absolute temperature. The actual relationship between \( V \) and \( I \) is investigated in this experiment by fitting obtained measurements of current and potential to 
	
	\papereq{bulb-model}{ V = k I^a }{Serbanescu}
	\begin{paperwhere}
		\papervar{V}{potential}{\volt}
		\papervar{I}{current}{\ampere}
		\papervar{k}{constant}{}
		\papervar{a}{costant}{}
	\end{paperwhere}

	Potential and current are measured using two multimeters, a power supply, a tungsten filament light bulb, and a switch arranged as follows:
	
	\paperfig{Setup}{\pdf{setup2}}{The experimental setup (light bulb).}
	
\papersec{Observations}

	\paperfig{Hydrogen}{\begin{papertable}{|[outer]I|[inner]C|[inner]C|[outer]}
			\paperoline\papertableindexheader&\textsc{Scale}&\multirow{2}{*}{\textsc{Colour}}\\
			\papertablecheaderunit{.02}{\hspace{-0.5ex}}&\\\paperiline
			\papertableindex\papertablecval{7.26}\papertablecval{Red}\\\paperiline
			\papertableindex\papertablecval{11.80}\papertablecval{Blue-Green}\\\paperiline
			\papertableindex\papertablecval{15.15}\papertablecval{Violet}\\\paperiline
			\papertableindex\papertablecval{17.75}\papertablecval{Violet}\\\paperoline
		\end{papertable}\vspace{-1.5em}}
	{The scale readings and colours of the spectral lines of hydrogen.}\vspace{1em}

\papersec{Analysis}

	Analysis

\papersec{Conclusion}

	Conclusion

\papersec{Sources}

	Sources

\papersource{}

\end{paper}

\end{document}