\documentclass[twoside]{article}
\usepackage{amsmath}
\usepackage{amssymb}
\usepackage{amsthm}
\usepackage{calc}
\usepackage{capt-of}
\usepackage{caption}
\usepackage[strict]{changepage}
\usepackage{chngcntr}
\usepackage[americanvoltage,siunitx]{circuitikz}
\usepackage{color,colortbl}
\usepackage{etoolbox}
\usepackage{fancyhdr}
\usepackage[T1]{fontenc}
\usepackage{gensymb}
\usepackage[margin=1in]{geometry}
\usepackage{graphicx}
\usepackage{hyperref}
\usepackage{import}
\usepackage{indentfirst}
\usepackage{mathptmx}
\usepackage{mathrsfs}
\usepackage{multicol}
\usepackage{multirow}
\usepackage{needspace}
\usepackage{pgfplots}
\usepackage{pgfplotstable}
\usepackage{setspace}
\usepackage{siunitx}
\usepackage{tabu}
\usepackage{tabularx}
\usepackage{tikz}
\usepackage{xspace}

\patchcmd{\thebibliography}{\section*{\refname}}{\vspace{-1em}}{}{}

\captionsetup{labelformat=empty,labelsep=none}
\usepgfplotslibrary{external}
\usetikzlibrary{positioning,matrix,shapes,chains,arrows}
\tikzexternalize[prefix=precompiled_figures/]

\newcommand\svgsize[2]{\def\svgwidth{#2}
{\centering\input{#1.pdf_tex}}}
\newcommand\svgc[1]{\svgsize{#1}{\columnwidth}}
\newcommand\svgl[1]{\svgsize{#1}{1em}}
\newcommand\diagrams[0]{\renewcommand\svgsize[2]{\def\svgwidth{##2}
{\centering\input{diagrams/##1.pdf_tex}}}}

\newcommand\pdf[1]{\noindent\includegraphics[width=\columnwidth]{#1.pdf}}
\newcommand\pdfex[1]{\pdf{#1}

\pdf{#1ex}}
\newcommand\pdfmsg[1]{\noindent\begin{minipage}{\columnwidth}\pdf{#1msg}

\pdf{#1}\end{minipage}}
\newcommand\pdfmsgex[1]{\pdfmsg{#1}

\pdf{#1ex}}
\newcommand\code[0]{\renewcommand\pdf[1]{\noindent
\includegraphics[width=\columnwidth]{code/##1.pdf}}}

% Indent
\setlength{\parindent}{0.3in}

\newcounter{paperthmamount}
\newcommand\theorems[0]{
\theoremstyle{remark}
\newtheorem{claim}[subsection]{Claim}
\theoremstyle{plain}
\newtheorem{conjecture}[subsection]{Conjecture}
\theoremstyle{plain}
\newtheorem{corollary}[subsection]{Corollary}
\theoremstyle{definition}
\newtheorem{definition}[subsection]{Definition}
\theoremstyle{plain}
\newtheorem{lemma}[subsection]{Lemma}
\theoremstyle{remark}
\newtheorem{proposition}[subsection]{Proposition}
\theoremstyle{remark}
\newtheorem{remark}[subsection]{Remark}
\theoremstyle{plain}
\newtheorem{theorem}[subsection]{Theorem}
\theoremstyle{definition}
\newtheorem{question}[subsection]{Question}
\newcommand\paperclm[2]
{\begin{claim}\global\expandafter\edef
\csname clm##1\endcsname{Claim \thesubsection\noexpand\xspace}
##2\end{claim}}
\newcommand\papercnj[2]
{\begin{conjecture}\global\expandafter\edef
\csname cnj##1\endcsname{Conjecture \thesubsection\noexpand\xspace}
##2\end{conjecture}}
\newcommand\papercor[2]
{\begin{corollary}\global\expandafter\edef
\csname cor##1\endcsname{Corollary \thesubsection\noexpand\xspace}
##2\end{corollary}}
\newcommand\paperdef[2]
{\begin{definition}\global\expandafter\edef
\csname def##1\endcsname{Definition \thesubsection\noexpand\xspace}
##2\end{definition}}
\newcommand\paperlem[2]
{\begin{lemma}\global\expandafter\edef
\csname lem##1\endcsname{Lemma \thesubsection\noexpand\xspace}
##2\end{lemma}}
\newcommand\paperprp[2]
{\begin{proposition}\global\expandafter\edef
\csname prp##1\endcsname{Proposition \thesubsection\noexpand\xspace}
##2\end{proposition}}
\newcommand\paperqtn[2]
{\begin{question}\global\expandafter\edef
\csname qtn##1\endcsname{Question \thesubsection\noexpand\xspace}
##2\end{question}}
\newcommand\paperrem[2]
{\begin{remark}\global\expandafter\edef
\csname rem##1\endcsname{Remark \thesubsection\noexpand\xspace}
##2\end{remark}}
\newcommand\paperthm[2]
{\begin{theorem}\global\expandafter\edef
\csname thm##1\endcsname{Theorem \thesubsection\noexpand\xspace}
##2\end{theorem}}}
\newcommand\subtheorems[0]{\stepcounter{paperthmamount}
\theoremstyle{remark}
\newtheorem{claim}[subsubsection]{Claim}
\theoremstyle{plain}
\newtheorem{conjecture}[subsubsection]{Conjecture}
\theoremstyle{plain}
\newtheorem{corollary}[subsubsection]{Corollary}
\theoremstyle{definition}
\newtheorem{definition}[subsubsection]{Definition}
\theoremstyle{plain}
\newtheorem{lemma}[subsubsection]{Lemma}
\theoremstyle{remark}
\newtheorem{proposition}[subsubsection]{Proposition}
\theoremstyle{remark}
\newtheorem{remark}[subsubsection]{Remark}
\theoremstyle{plain}
\newtheorem{theorem}[subsubsection]{Theorem}
\theoremstyle{definition}
\newtheorem{question}[subsubsection]{Question}
\newcommand\paperclm[2]
{\begin{claim}\global\expandafter\edef
\csname clm##1\endcsname{Claim \thesubsubsection\noexpand\xspace}
##2\end{claim}}
\newcommand\papercnj[2]
{\begin{conjecture}\global\expandafter\edef
\csname cnj##1\endcsname{Conjecture \thesubsubsection\noexpand\xspace}
##2\end{conjecture}}
\newcommand\papercor[2]
{\begin{corollary}\global\expandafter\edef
\csname cor##1\endcsname{Corollary \thesubsubsection\noexpand\xspace}
##2\end{corollary}}
\newcommand\paperdef[2]
{\begin{definition}\global\expandafter\edef
\csname def##1\endcsname{Definition \thesubsubsection\noexpand\xspace}
##2\end{definition}}
\newcommand\paperlem[2]
{\begin{lemma}\global\expandafter\edef
\csname lem##1\endcsname{Lemma \thesubsubsection\noexpand\xspace}
##2\end{lemma}}
\newcommand\paperprp[2]
{\begin{proposition}\global\expandafter\edef
\csname prp##1\endcsname{Proposition \thesubsubsection\noexpand\xspace}
##2\end{proposition}}
\newcommand\paperqtn[2]
{\begin{question}\global\expandafter\edef
\csname qtn##1\endcsname{Question \thesubsubsection\noexpand\xspace}
##2\end{question}}
\newcommand\paperrem[2]
{\begin{remark}\global\expandafter\edef
\csname rem##1\endcsname{Remark \thesubsubsection\noexpand\xspace}
##2\end{remark}}
\newcommand\paperthm[2]
{\begin{theorem}\global\expandafter\edef
\csname thm##1\endcsname{Theorem \thesubsubsection\noexpand\xspace}
##2\end{theorem}}}

% Title section
\pagestyle{fancy}
\thispagestyle{empty}
\renewcommand{\headrulewidth}{0pt}
\newcommand\papertitle[1]
{{\centering\fontsize{20pt}{20pt}\textsc{#1}\\\mbox{}\\}
\fancyhead[OC]{\fontsize{12pt}{12pt}\selectfont\textit{#1}}}
\newcounter{people}
\newcommand\paperauthtext[3]{{\centering\fontsize{12pt}{12pt}\selectfont
\textsc{#1}\\[-0.1em]{\fontsize{9pt}{9pt}\selectfont\textit{\ifx&#2&
\vspace{-1em}\else#2\fi}}\\\mbox{}\\
\fancyhead[EC]{\fontsize{12pt}{12pt}\selectfont\textit{#3}}}}
\newcommand\paperauth[2]{{\stepcounter{people}
\ifnum\value{people}=1
{\paperauthtext{#1}{#2}{#1}
\global\def\auth{#1\xspace}}
\else\ifnum\value{people}=2
{\paperauthtext{#1}{#2}{\auth and #1}}
\else{\paperauthtext{#1}{#2}{\auth et al}}\fi\fi}}
\newcommand\physics[0]{
\renewcommand\paperauthtext[4]{{\centering\fontsize{12pt}{12pt}\selectfont
\textsc{##1. ##2}\\[-0.1em]{\fontsize{9pt}{9pt}\selectfont\textit{\ifx&##3&
\vspace{-1em}\else##3\fi}}\\\mbox{}\\
\fancyhead[EC]{\fontsize{12pt}{12pt}\selectfont\textit{##4}}}}
\renewcommand\paperauth[3]{{\stepcounter{people}
\ifnum\value{people}=1
{\paperauthtext{##1}{##2}{##3}{##1. ##2}
\global\def\auth{##2\xspace}}
\else\ifnum\value{people}=2
{\paperauthtext{##1}{##2}{##3}{\auth and ##2}}
\else{\paperauthtext{##1}{##2}{##3}{\auth et al}}\fi\fi}}}
\newcommand\paperdate[1]{{\centering\fontsize{9pt}{9pt}\selectfont\text{
(Received #1)}\\[2em]}}

% Page header
\newcommand{\paperhead}[1]{\fancyhead[EC]{\fontsize{12pt}{12pt}\selectfont
\textit{#1}}}
\fancyhead[RO, EL]{\fontsize{12pt}{12pt}\selectfont\thepage}
\fancyhead[RE, OL]{}
\cfoot{}

\makeatletter
\newenvironment{paperadjustwidth}[2]{
  \begin{list}{}{
    \setlength\partopsep\z@
    \setlength\topsep\z@
    \setlength\listparindent\parindent
    \setlength\parsep\parskip
    \@ifmtarg{#1}{\setlength{\leftmargin}{\z@}}
                 {\setlength{\leftmargin}{#1}}
    \@ifmtarg{#2}{\setlength{\rightmargin}{\z@}}
                 {\setlength{\rightmargin}{#2}}
    }
    \item[]}{\end{list}}
\makeatother

%Figure counter
\newcounter{paperfigurecounter}
\newcommand{\papercap}[2]{\bgroup\stepcounter{paperfigurecounter}
\captionof{figure}{\fontsize{9pt}{9pt}\selectfont
\hspace{0.3in}Fig.~\arabic{paperfigurecounter}.\quad#2}
\egroup\expandafter\edef
\csname fig#1\endcsname{Fig.~\arabic{paperfigurecounter}\noexpand\xspace}}

\newcommand\paperfig[3]{\noindent\begin{minipage}{\columnwidth}
#2\papercap{#1}{#3}\end{minipage}\expandafter\edef
\csname fig#1\endcsname{Fig.~\arabic{paperfigurecounter}\noexpand\xspace}}
\newcommand\papersvg[3]{\paperfig{#1}{\svgc{#2}}{#3}}

% Abstract environment
\newenvironment{paperabs}
{\begin{paperadjustwidth}{0.5in}{0.5in}\bgroup\fontsize{9pt}{9pt}\selectfont
\hspace{0.5in}}
{\egroup\end{paperadjustwidth}}

% Paper environment
\setlength\columnsep{0.5in}
\newenvironment{paper}
{\begin{multicols*}{2}\bgroup\fontsize{12pt}{12pt}\selectfont}
{\egroup\end{multicols*}}
\newcommand{\singlecolumn}[0]{
\renewcommand\paperfig[3]{\noindent
\makebox[\textwidth][c]{\begin{minipage}{5.5in}
\noindent\makebox[\textwidth][c]{\begin{minipage}{3in}##2\end{minipage}}
\papercap{##1}{##3}\end{minipage}}\expandafter\edef
\csname fig##1\endcsname{Fig.~\arabic{paperfigurecounter}\noexpand\xspace}}
\renewenvironment{paper}{\bgroup\fontsize{12pt}{12pt}\selectfont}
{\egroup}}

%Sources
\newsavebox{\sourcebox}
\newcommand{\papersource}[1]{
\vspace{-2em}
\text{}\\*
\fontsize{9pt}{9pt}\selectfont
\noindent\renewcommand{\labelenumi}{}
\savebox{\sourcebox}{\parbox{3in}{\begin{enumerate}
\setlength{\leftmargini}{-1ex}
\setlength{\leftmargin}{-1ex}
\setlength{\labelwidth}{0pt}
\setlength{\labelsep}{0pt}
\setlength{\listparindent}{0pt}
\item\textit{\hspace{-0.35in}#1}
\end{enumerate}}}
\usebox{\sourcebox}
}

%Section headers
\newcounter{paperseccounter}
\newcounter{papersubseccounter}[paperseccounter]
\newcommand\papersec[1]{\needspace{1in}
\stepcounter{paperseccounter}
\stepcounter{section}
\begin{center}\Roman{paperseccounter} \textsc{#1}\end{center}}
\newcommand\papersubsec[1]{\needspace{1in}
\stepcounter{papersubseccounter}
\addtocounter{subsection}{\thepaperthmamount}
\setcounter{subsubsection}{0}
{\begin{center}
\Roman{section}.\Roman{papersubseccounter}
\textsc{#1}\\[0.5em]\end{center}}}

%equation
\newcounter{papereqcounter}
\newcommand\papereq[3]{{
\stepcounter{papereqcounter}
\mbox{}\vspace{-0.75em}
\begin{equation*}
#2
\tag*{\fontsize{12pt}{12pt}\selectfont
$\begin{array}{r}
\cr{\text{(\arabic{papereqcounter})}}
\cr{\fontsize{9pt}{9pt}\selectfont\textit{\ifx\\#3\\~\else(\fi#3\ifx\\#3\\~
\else)\fi}}
\end{array}$}
\end{equation*}

}
\expandafter\edef\csname eq#1\endcsname{(\arabic{papereqcounter})\noexpand
\xspace}}

% Where
\newcommand{\papervar}[3]
{&$#1$ & #2 \ifx\\#3\\~\else($\smash{\text{\si{\fi
#3\ifx\\#3\\~\else}}}$)\fi\\}
\newenvironment{paperwhere}
{\begin{minipage}{\columnwidth}
\bgroup\fontsize{9pt}{9pt}\selectfont Where:\vspace{2pt}\\\begin{tabular}
{rr@{ = }p{\linewidth}}}
{\end{tabular}\egroup\end{minipage}\vspace{5pt}}

% Tables
\definecolor{LineGray}{gray}{0.5}
\newtabulinestyle{outer=2.25pt LineGray}
\newtabulinestyle{inner=0.75pt LineGray}
\tabulinesep=1.5pt

\newcommand{\paperiline}[0]{\tabucline[inner]{-}}
\newcommand{\paperoline}[0]{\tabucline[outer]{-}}

% Index column type
\newcolumntype{I}{X[-5,c]}
% Column type with uncertainty
\newcolumntype{U}{@{}X[-5,r]@{$\pm$}X[-5,l]@{}}
% Column type without uncertainty
\newcolumntype{C}{@{}X[-5,c]@{}}

\newcounter{papertableindexcounter}
\newcommand{\papertableindexheader}[0]{\multirow{2}{*}{\textsc{Index}}}
\newcommand{\papertableindex}[0]{\stepcounter{papertableindexcounter}
\arabic{papertableindexcounter}}
\newcommand{\papertableuheadersymbol}[1]{&\multicolumn{2}{c|[inner]}{$#1$}}
\newcommand{\papertableuheadersymbole}[1]{&\multicolumn{2}{c|[outer]}{$#1$}}
\newcommand{\papertableuheaderunit}[1]{&\multicolumn{2}{c|[inner]}{(#1)}}
\newcommand{\papertableuheaderunite}[1]{&\multicolumn{2}{c|[outer]}{(#1)}}
\newcommand{\papertablecheadersymbol}[1]{&$#1$}
\newcommand{\papertablecheaderunit}[2]{&($\pm$#1 #2)}

% Value in table with uncertainty.
\newcommand{\papertableuval}[2]{& #1 & #2}
% Value in table without uncertainty.
\newcommand{\papertablecval}[1]{& #1}

\newenvironment{papertable}[1]
{\setcounter{papertableindexcounter}{0} 
\begin{tabu} to \linewidth {#1}}
{\end{tabu}\vspace{12pt}}

\newcommand{\paperaxis}[9]
{title=#1,
axis x line = bottom,
xmin=#4,xmax=#6,
axis y line = left,
ymin=#5,ymax=#7,
height = 180pt,
grid=both,
x axis line style=-,
y axis line style=-,
x tick label style={
/pgf/number format/.cd,
fixed,
fixed zerofill,
precision=#8,
/tikz/.cd},
y tick label style={
/pgf/number format/.cd,
fixed,
fixed zerofill,
precision=#9,
/tikz/.cd}}
\newcommand{\paperaxisxlabel}[2]{
xlabel=\fontsize{10pt}{10pt}\selectfont#1$(#2)\rightarrow$}
\newcommand{\paperaxisylabel}[2]{
ylabel=\fontsize{10pt}{10pt}\selectfont#1$(#2)\rightarrow$}
\newcommand{\papergraphoutline}[4]{
\addplot [mark=none,line width=0.75pt] coordinates {
(#1,#2)
(#1,#4)
(#3,#4)
(#3,#2)
(#1,#2)};}

\newenvironment{papergraph}{
\begin{tikzpicture}
\begin{axis}}
{\end{axis}
\end{tikzpicture}}

\newcommand{\comment}[1]{}

\newcommand{\abs}[1]{\left\lvert#1\right\rvert}
\newcommand{\oo}[0]{\infty}
\newcommand{\sigmaSum}[3]{\sum\limits_{#1}^{#2} #3}
\newcommand{\limto}[3]{\lim\limits_{#1\rightarrow#2}#3}
\renewcommand{\d}[0]{\mathrm{d}}
\newcommand{\cross}[0]{\times}
\newcommand{\lp}{\left(}
\newcommand{\rp}{\right)}
\newcommand\pars[1]{\lp#1\rp}
\newcommand\sqbrack[1]{\left[#1\right]}
\newcommand\R{\mathbb{R}}
\newcommand\di{\partial}
\newcommand\x{\times}
\newcommand\del{\nabla}

\physics
\begin{document}
\papertitle{Non-linear Circuit Elements}
\paperauth{A}{Khesin}{1002442029}
\paperauth{P}{Zavyalova}{1002345036}
\paperdate{January 17, 2018}
\begin{paperabs}
	
	Abstract goes here
	
\end{paperabs}

\begin{paper}
	
\papersec{Introduction}

	The purpose of this experiment was to study the relationship between voltage and current for a thermistor and an incandescent lamp. Both of these circuit elements are nonlinear; for a thermistor, the resistance is a function of temperature and obeys the relation
	
	\papereq{Thermistor}{R = R_0e^{ -\frac{T}{T_0} }}{Serbanescu}
	\begin{paperwhere}
		\papervar{T}{absolute temperature}{\kelvin}
		\papervar{T_0}{constant of device}{\kelvin}
		\papervar{R_0}{constant of device}{\ohm}
	\end{paperwhere}

	The relationship was investigated using a multimeter, a thermistor, a thermometer and a thermal flask for maintaining the thermistor at different temperatures, set up as shown:
	
	\paperfig{Setup}{\pdf{setup1}}{The experimental setup (thermistor).}
	
	The temperature of the thermistor was decreased by adding cold water to the thermal flask. 
	
	The tungsten filament in a light bulb follows Ohm's law, 
	
	\papereq{Ohm}{ R = \frac{V}{I} }{Serbanescu}
	\begin{paperwhere}
		\papervar{R}{resistance}{\ohm}
		\papervar{V}{potential}{\volt}
		\papervar{I}{current}{\ampere}
	\end{paperwhere}
	
	However, as the temperature increases, the resistance of the metal increases and above a certain temperature becomes proportional to the absolute temperature. The actual relationship between \( V \) and \( I \) is investigated in this experiment by fitting obtained measurements of current and potential to 
	
	\papereq{bulb-model}{ V = k I^a }{Serbanescu}
	\begin{paperwhere}
		\papervar{V}{potential}{\volt}
		\papervar{I}{current}{\ampere}
		\papervar{k}{constant}{}
		\papervar{a}{costant}{}
	\end{paperwhere}

	Potential and current are measured using two multimeters, a power supply, a tungsten filament light bulb, and a switch arranged as follows:
	
	\paperfig{Setup}{\pdf{setup2}}{The experimental setup (light bulb).}
	
\papersec{Observations}

	The resistance of the thermistor at various temperatures was obtained and recorded. The uncertainty in resistance is taken to be the last digit increment in the least significant digit displayed.
	
	\paperfig{Thermistor}{\begin{papertable}{|[outer]I|[inner]C|[inner]U|[outer]}
			\paperoline\papertableindexheader&\textsc{Temperature}&\multicolumn{2}{c|[outer]}{\textsc{Resistance}}\\
			\papertablecheadersymbol{T$ $(\pm0.5$ $\si{\kelvin})}\papertableuheadersymbole{R$ $(\si{\kilo\ohm})}\\\paperiline
			\papertableindex\papertablecval{352.5}\papertableuval{6.68}{.02}\\\paperiline
			\papertableindex\papertablecval{349.0}\papertableuval{6.93}{.02}\\\paperiline
			\papertableindex\papertablecval{348.0}\papertableuval{7.54}{.03}\\\paperiline
			\papertableindex\papertablecval{346.0}\papertableuval{8.01}{.03}\\\paperiline
			\papertableindex\papertablecval{340.0}\papertableuval{9.95}{.03}\\\paperiline
			\papertableindex\papertablecval{338.0}\papertableuval{10.33}{.03}\\\paperiline
			\papertableindex\papertablecval{335.0}\papertableuval{11.76}{.03}\\\paperiline
			\papertableindex\papertablecval{334.5}\papertableuval{12.36}{.03}\\\paperiline
			\papertableindex\papertablecval{329.5}\papertableuval{14.45}{.04}\\\paperiline
			\papertableindex\papertablecval{327.5}\papertableuval{15.33}{.04}\\\paperiline
			\papertableindex\papertablecval{324.0}\papertableuval{17.23}{.04}\\\paperiline
			\papertableindex\papertablecval{321.0}\papertableuval{19.51}{.05}\\\paperiline
			\papertableindex\papertablecval{319.0}\papertableuval{21.7}{.1}\\\paperiline
			\papertableindex\papertablecval{317.0}\papertableuval{24.3}{.1}\\\paperiline
			\papertableindex\papertablecval{314.0}\papertableuval{26.1}{.2}\\\paperiline
			\papertableindex\papertablecval{309.5}\papertableuval{31.4}{.2}\\\paperiline
			\papertableindex\papertablecval{308.5}\papertableuval{33.7}{.2}\\\paperiline
			\papertableindex\papertablecval{307.5}\papertableuval{35.5}{.2}\\\paperiline
			\papertableindex\papertablecval{305.0}\papertableuval{38.0}{.2}\\\paperiline
			\papertableindex\papertablecval{304.0}\papertableuval{40.3}{.2}\\\paperiline
			\papertableindex\papertablecval{301.5}\papertableuval{44.3}{.2}\\\paperiline
			\papertableindex\papertablecval{299.5}\papertableuval{47.7}{.2}\\\paperiline
			\papertableindex\papertablecval{298.0}\papertableuval{51.6}{.2}\\\paperiline
			\papertableindex\papertablecval{296.0}\papertableuval{56.3}{.2}\\\paperiline
			\papertableindex\papertablecval{294.0}\papertableuval{61.3}{.2}\\\paperiline
			\papertableindex\papertablecval{292.5}\papertableuval{66.9}{.2}\\\paperiline
			\papertableindex\papertablecval{290.5}\papertableuval{73.8}{.2}\\\paperiline
			\papertableindex\papertablecval{286.0}\papertableuval{92.9}{.3}\\\paperiline
			\papertableindex\papertablecval{283.0}\papertableuval{104.7}{.3}\\\paperiline
			\papertableindex\papertablecval{278.0}\papertableuval{~126.2}{\parbox{\widthof{126.2~}}{.4}}\\\paperoline
		\end{papertable}\vspace{-1.5em}}
	{Resistance of thermistor at the given temperatures.}\vspace{1em}

	Further, current and potential for a tungsten filament light bulb were measured and recorded. The uncertainty is again taken to be the last digit increment in the least significant digit displayed. 

	\paperfig{Light bulb}{\begin{papertable}{|[outer]I|[inner]U|[inner]U|[outer]}
			\paperoline\papertableindexheader&\multicolumn{2}{c|[inner]}{\textsc{Potential}}&\multicolumn{2}{c|[outer]}{\textsc{Current}}\\
			\papertableuheadersymbol{V$ $(\si{\volt})}\papertableuheadersymbole{I$ $(\si{\milli\ampere)}}\\\paperiline
			\papertableindex\papertableuval{0.0222}{.0002}\papertableuval{0.324}{.003}\\\paperiline
			\papertableindex\papertableuval{0.0776}{.0003}\papertableuval{1.045}{.009}\\\paperiline
			\papertableindex\papertableuval{0.1122}{.0004}\papertableuval{1.39}{.01}\\\paperiline
			\papertableindex\papertableuval{0.244}{.002}\papertableuval{2.27}{.03}\\\paperiline
			\papertableindex\papertableuval{0.472}{.002}\papertableuval{3.40}{.04}\\\paperiline
			\papertableindex\papertableuval{0.541}{.002}\papertableuval{3.69}{.04}\\\paperiline
			\papertableindex\papertableuval{0.697}{.003}\papertableuval{4.28}{.04}\\\paperiline
			\papertableindex\papertableuval{0.871}{.003}\papertableuval{4.89}{.05}\\\paperiline
			\papertableindex\papertableuval{1.128}{.004}\papertableuval{5.72}{.05}\\\paperiline
			\papertableindex\papertableuval{1.840}{.006}\papertableuval{7.68}{.07}\\\paperiline
			\papertableindex\papertableuval{1.940}{.006}\papertableuval{7.93}{.07}\\\paperiline
			\papertableindex\papertableuval{4.01}{.02}\papertableuval{12.3}{.1}\\\paperiline
			\papertableindex\papertableuval{6.00}{.03}\papertableuval{15.8}{.1}\\\paperiline
			\papertableindex\papertableuval{8.01}{.03}\papertableuval{18.8}{.2}\\\paperiline
			\papertableindex\papertableuval{10.24}{.04}\papertableuval{21.8}{.3}\\\paperiline
			\papertableindex\papertableuval{12.07}{.04}\papertableuval{24.1}{.3}\\\paperiline
			\papertableindex\papertableuval{14.27}{.05}\papertableuval{26.6}{.3}\\\paperiline
			\papertableindex\papertableuval{16.01}{.05}\papertableuval{28.5}{.3}\\\paperiline
			\papertableindex\papertableuval{17.98}{.05}\papertableuval{30.6}{.3}\\\paperiline
			\papertableindex\papertableuval{18.45}{.06}\papertableuval{31.1}{.3}\\\paperiline
			\papertableindex\papertableuval{19.86}{.06}\papertableuval{32.5}{.3}\\\paperiline
			\papertableindex\papertableuval{21.0}{.2}\papertableuval{33.6}{.4}\\\paperiline
			\papertableindex\papertableuval{22.1}{.2}\papertableuval{34.6}{.4}\\\paperiline
			\papertableindex\papertableuval{22.9}{.2}\papertableuval{35.4}{.4}\\\paperiline
			\papertableindex\papertableuval{24.0}{.2}\papertableuval{36.4}{.4}\\\paperiline
			\papertableindex\papertableuval{25.1}{.2}\papertableuval{37.3}{.4}\\\paperoline
			\end{papertable}\vspace{-1.5em}}
	{Potential and current through the light bulb.}\vspace{1em}

\papersec{Analysis}

	Analysis

\papersec{Conclusion}

	Conclusion

\papersec{Sources}

	Sources

\papersource{}

\end{paper}

\end{document}