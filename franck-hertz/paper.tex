\documentclass[twoside]{article}
\usepackage{amsmath}
\usepackage{amssymb}
\usepackage{amsthm}
\usepackage{calc}
\usepackage{capt-of}
\usepackage{caption}
\usepackage[strict]{changepage}
\usepackage{chngcntr}
\usepackage[americanvoltage,siunitx]{circuitikz}
\usepackage{color,colortbl}
\usepackage{etoolbox}
\usepackage{fancyhdr}
\usepackage[T1]{fontenc}
\usepackage{gensymb}
\usepackage[margin=1in]{geometry}
\usepackage{graphicx}
\usepackage{hyperref}
\usepackage{import}
\usepackage{indentfirst}
\usepackage{mathptmx}
\usepackage{mathrsfs}
\usepackage{multicol}
\usepackage{multirow}
\usepackage{needspace}
\usepackage{pgfplots}
\usepackage{pgfplotstable}
\usepackage{setspace}
\usepackage{siunitx}
\usepackage{tabu}
\usepackage{tabularx}
\usepackage{tikz}
\usepackage{xspace}

\patchcmd{\thebibliography}{\section*{\refname}}{\vspace{-1em}}{}{}

\captionsetup{labelformat=empty,labelsep=none}
\usepgfplotslibrary{external}
\usetikzlibrary{positioning,matrix,shapes,chains,arrows}
\tikzexternalize[prefix=precompiled_figures/]

\newcommand\svgsize[2]{\def\svgwidth{#2}
{\centering\input{#1.pdf_tex}}}
\newcommand\svgc[1]{\svgsize{#1}{\columnwidth}}
\newcommand\svgl[1]{\svgsize{#1}{1em}}
\newcommand\diagrams[0]{\renewcommand\svgsize[2]{\def\svgwidth{##2}
{\centering\input{diagrams/##1.pdf_tex}}}}

\newcommand\pdf[1]{\noindent\includegraphics[width=\columnwidth]{#1.pdf}}
\newcommand\pdfex[1]{\pdf{#1}

\pdf{#1ex}}
\newcommand\pdfmsg[1]{\noindent\begin{minipage}{\columnwidth}\pdf{#1msg}

\pdf{#1}\end{minipage}}
\newcommand\pdfmsgex[1]{\pdfmsg{#1}

\pdf{#1ex}}
\newcommand\code[0]{\renewcommand\pdf[1]{\noindent
\includegraphics[width=\columnwidth]{code/##1.pdf}}}

% Indent
\setlength{\parindent}{0.3in}

\newcounter{paperthmamount}
\newcommand\theorems[0]{
\theoremstyle{remark}
\newtheorem{claim}[subsection]{Claim}
\theoremstyle{plain}
\newtheorem{conjecture}[subsection]{Conjecture}
\theoremstyle{plain}
\newtheorem{corollary}[subsection]{Corollary}
\theoremstyle{definition}
\newtheorem{definition}[subsection]{Definition}
\theoremstyle{plain}
\newtheorem{lemma}[subsection]{Lemma}
\theoremstyle{remark}
\newtheorem{proposition}[subsection]{Proposition}
\theoremstyle{remark}
\newtheorem{remark}[subsection]{Remark}
\theoremstyle{plain}
\newtheorem{theorem}[subsection]{Theorem}
\theoremstyle{definition}
\newtheorem{question}[subsection]{Question}
\newcommand\paperclm[2]
{\begin{claim}\global\expandafter\edef
\csname clm##1\endcsname{Claim \thesubsection\noexpand\xspace}
##2\end{claim}}
\newcommand\papercnj[2]
{\begin{conjecture}\global\expandafter\edef
\csname cnj##1\endcsname{Conjecture \thesubsection\noexpand\xspace}
##2\end{conjecture}}
\newcommand\papercor[2]
{\begin{corollary}\global\expandafter\edef
\csname cor##1\endcsname{Corollary \thesubsection\noexpand\xspace}
##2\end{corollary}}
\newcommand\paperdef[2]
{\begin{definition}\global\expandafter\edef
\csname def##1\endcsname{Definition \thesubsection\noexpand\xspace}
##2\end{definition}}
\newcommand\paperlem[2]
{\begin{lemma}\global\expandafter\edef
\csname lem##1\endcsname{Lemma \thesubsection\noexpand\xspace}
##2\end{lemma}}
\newcommand\paperprp[2]
{\begin{proposition}\global\expandafter\edef
\csname prp##1\endcsname{Proposition \thesubsection\noexpand\xspace}
##2\end{proposition}}
\newcommand\paperqtn[2]
{\begin{question}\global\expandafter\edef
\csname qtn##1\endcsname{Question \thesubsection\noexpand\xspace}
##2\end{question}}
\newcommand\paperrem[2]
{\begin{remark}\global\expandafter\edef
\csname rem##1\endcsname{Remark \thesubsection\noexpand\xspace}
##2\end{remark}}
\newcommand\paperthm[2]
{\begin{theorem}\global\expandafter\edef
\csname thm##1\endcsname{Theorem \thesubsection\noexpand\xspace}
##2\end{theorem}}}
\newcommand\subtheorems[0]{\stepcounter{paperthmamount}
\theoremstyle{remark}
\newtheorem{claim}[subsubsection]{Claim}
\theoremstyle{plain}
\newtheorem{conjecture}[subsubsection]{Conjecture}
\theoremstyle{plain}
\newtheorem{corollary}[subsubsection]{Corollary}
\theoremstyle{definition}
\newtheorem{definition}[subsubsection]{Definition}
\theoremstyle{plain}
\newtheorem{lemma}[subsubsection]{Lemma}
\theoremstyle{remark}
\newtheorem{proposition}[subsubsection]{Proposition}
\theoremstyle{remark}
\newtheorem{remark}[subsubsection]{Remark}
\theoremstyle{plain}
\newtheorem{theorem}[subsubsection]{Theorem}
\theoremstyle{definition}
\newtheorem{question}[subsubsection]{Question}
\newcommand\paperclm[2]
{\begin{claim}\global\expandafter\edef
\csname clm##1\endcsname{Claim \thesubsubsection\noexpand\xspace}
##2\end{claim}}
\newcommand\papercnj[2]
{\begin{conjecture}\global\expandafter\edef
\csname cnj##1\endcsname{Conjecture \thesubsubsection\noexpand\xspace}
##2\end{conjecture}}
\newcommand\papercor[2]
{\begin{corollary}\global\expandafter\edef
\csname cor##1\endcsname{Corollary \thesubsubsection\noexpand\xspace}
##2\end{corollary}}
\newcommand\paperdef[2]
{\begin{definition}\global\expandafter\edef
\csname def##1\endcsname{Definition \thesubsubsection\noexpand\xspace}
##2\end{definition}}
\newcommand\paperlem[2]
{\begin{lemma}\global\expandafter\edef
\csname lem##1\endcsname{Lemma \thesubsubsection\noexpand\xspace}
##2\end{lemma}}
\newcommand\paperprp[2]
{\begin{proposition}\global\expandafter\edef
\csname prp##1\endcsname{Proposition \thesubsubsection\noexpand\xspace}
##2\end{proposition}}
\newcommand\paperqtn[2]
{\begin{question}\global\expandafter\edef
\csname qtn##1\endcsname{Question \thesubsubsection\noexpand\xspace}
##2\end{question}}
\newcommand\paperrem[2]
{\begin{remark}\global\expandafter\edef
\csname rem##1\endcsname{Remark \thesubsubsection\noexpand\xspace}
##2\end{remark}}
\newcommand\paperthm[2]
{\begin{theorem}\global\expandafter\edef
\csname thm##1\endcsname{Theorem \thesubsubsection\noexpand\xspace}
##2\end{theorem}}}

% Title section
\pagestyle{fancy}
\thispagestyle{empty}
\renewcommand{\headrulewidth}{0pt}
\newcommand\papertitle[1]
{{\centering\fontsize{20pt}{20pt}\textsc{#1}\\\mbox{}\\}
\fancyhead[OC]{\fontsize{12pt}{12pt}\selectfont\textit{#1}}}
\newcounter{people}
\newcommand\paperauthtext[3]{{\centering\fontsize{12pt}{12pt}\selectfont
\textsc{#1}\\[-0.1em]{\fontsize{9pt}{9pt}\selectfont\textit{\ifx&#2&
\vspace{-1em}\else#2\fi}}\\\mbox{}\\
\fancyhead[EC]{\fontsize{12pt}{12pt}\selectfont\textit{#3}}}}
\newcommand\paperauth[2]{{\stepcounter{people}
\ifnum\value{people}=1
{\paperauthtext{#1}{#2}{#1}
\global\def\auth{#1\xspace}}
\else\ifnum\value{people}=2
{\paperauthtext{#1}{#2}{\auth and #1}}
\else{\paperauthtext{#1}{#2}{\auth et al}}\fi\fi}}
\newcommand\physics[0]{
\renewcommand\paperauthtext[4]{{\centering\fontsize{12pt}{12pt}\selectfont
\textsc{##1. ##2}\\[-0.1em]{\fontsize{9pt}{9pt}\selectfont\textit{\ifx&##3&
\vspace{-1em}\else##3\fi}}\\\mbox{}\\
\fancyhead[EC]{\fontsize{12pt}{12pt}\selectfont\textit{##4}}}}
\renewcommand\paperauth[3]{{\stepcounter{people}
\ifnum\value{people}=1
{\paperauthtext{##1}{##2}{##3}{##1. ##2}
\global\def\auth{##2\xspace}}
\else\ifnum\value{people}=2
{\paperauthtext{##1}{##2}{##3}{\auth and ##2}}
\else{\paperauthtext{##1}{##2}{##3}{\auth et al}}\fi\fi}}}
\newcommand\paperdate[1]{{\centering\fontsize{9pt}{9pt}\selectfont\text{
(Received #1)}\\[2em]}}

% Page header
\newcommand{\paperhead}[1]{\fancyhead[EC]{\fontsize{12pt}{12pt}\selectfont
\textit{#1}}}
\fancyhead[RO, EL]{\fontsize{12pt}{12pt}\selectfont\thepage}
\fancyhead[RE, OL]{}
\cfoot{}

\makeatletter
\newenvironment{paperadjustwidth}[2]{
  \begin{list}{}{
    \setlength\partopsep\z@
    \setlength\topsep\z@
    \setlength\listparindent\parindent
    \setlength\parsep\parskip
    \@ifmtarg{#1}{\setlength{\leftmargin}{\z@}}
                 {\setlength{\leftmargin}{#1}}
    \@ifmtarg{#2}{\setlength{\rightmargin}{\z@}}
                 {\setlength{\rightmargin}{#2}}
    }
    \item[]}{\end{list}}
\makeatother

%Figure counter
\newcounter{paperfigurecounter}
\newcommand{\papercap}[2]{\bgroup\stepcounter{paperfigurecounter}
\captionof{figure}{\fontsize{9pt}{9pt}\selectfont
\hspace{0.3in}Fig.~\arabic{paperfigurecounter}.\quad#2}
\egroup\expandafter\edef
\csname fig#1\endcsname{Fig.~\arabic{paperfigurecounter}\noexpand\xspace}}

\newcommand\paperfig[3]{\noindent\begin{minipage}{\columnwidth}
#2\papercap{#1}{#3}\end{minipage}\expandafter\edef
\csname fig#1\endcsname{Fig.~\arabic{paperfigurecounter}\noexpand\xspace}}
\newcommand\papersvg[3]{\paperfig{#1}{\svgc{#2}}{#3}}

% Abstract environment
\newenvironment{paperabs}
{\begin{paperadjustwidth}{0.5in}{0.5in}\bgroup\fontsize{9pt}{9pt}\selectfont
\hspace{0.5in}}
{\egroup\end{paperadjustwidth}}

% Paper environment
\setlength\columnsep{0.5in}
\newenvironment{paper}
{\begin{multicols*}{2}\bgroup\fontsize{12pt}{12pt}\selectfont}
{\egroup\end{multicols*}}
\newcommand{\singlecolumn}[0]{
\renewcommand\paperfig[3]{\noindent
\makebox[\textwidth][c]{\begin{minipage}{5.5in}
\noindent\makebox[\textwidth][c]{\begin{minipage}{3in}##2\end{minipage}}
\papercap{##1}{##3}\end{minipage}}\expandafter\edef
\csname fig##1\endcsname{Fig.~\arabic{paperfigurecounter}\noexpand\xspace}}
\renewenvironment{paper}{\bgroup\fontsize{12pt}{12pt}\selectfont}
{\egroup}}

%Sources
\newsavebox{\sourcebox}
\newcommand{\papersource}[1]{
\vspace{-2em}
\text{}\\*
\fontsize{9pt}{9pt}\selectfont
\noindent\renewcommand{\labelenumi}{}
\savebox{\sourcebox}{\parbox{3in}{\begin{enumerate}
\setlength{\leftmargini}{-1ex}
\setlength{\leftmargin}{-1ex}
\setlength{\labelwidth}{0pt}
\setlength{\labelsep}{0pt}
\setlength{\listparindent}{0pt}
\item\textit{\hspace{-0.35in}#1}
\end{enumerate}}}
\usebox{\sourcebox}
}

%Section headers
\newcounter{paperseccounter}
\newcounter{papersubseccounter}[paperseccounter]
\newcommand\papersec[1]{\needspace{1in}
\stepcounter{paperseccounter}
\stepcounter{section}
\begin{center}\Roman{paperseccounter} \textsc{#1}\end{center}}
\newcommand\papersubsec[1]{\needspace{1in}
\stepcounter{papersubseccounter}
\addtocounter{subsection}{\thepaperthmamount}
\setcounter{subsubsection}{0}
{\begin{center}
\Roman{section}.\Roman{papersubseccounter}
\textsc{#1}\\[0.5em]\end{center}}}

%equation
\newcounter{papereqcounter}
\newcommand\papereq[3]{{
\stepcounter{papereqcounter}
\mbox{}\vspace{-0.75em}
\begin{equation*}
#2
\tag*{\fontsize{12pt}{12pt}\selectfont
$\begin{array}{r}
\cr{\text{(\arabic{papereqcounter})}}
\cr{\fontsize{9pt}{9pt}\selectfont\textit{\ifx\\#3\\~\else(\fi#3\ifx\\#3\\~
\else)\fi}}
\end{array}$}
\end{equation*}

}
\expandafter\edef\csname eq#1\endcsname{(\arabic{papereqcounter})\noexpand
\xspace}}

% Where
\newcommand{\papervar}[3]
{&$#1$ & #2 \ifx\\#3\\~\else($\smash{\text{\si{\fi
#3\ifx\\#3\\~\else}}}$)\fi\\}
\newenvironment{paperwhere}
{\begin{minipage}{\columnwidth}
\bgroup\fontsize{9pt}{9pt}\selectfont Where:\vspace{2pt}\\\begin{tabular}
{rr@{ = }p{\linewidth}}}
{\end{tabular}\egroup\end{minipage}\vspace{5pt}}

% Tables
\definecolor{LineGray}{gray}{0.5}
\newtabulinestyle{outer=2.25pt LineGray}
\newtabulinestyle{inner=0.75pt LineGray}
\tabulinesep=1.5pt

\newcommand{\paperiline}[0]{\tabucline[inner]{-}}
\newcommand{\paperoline}[0]{\tabucline[outer]{-}}

% Index column type
\newcolumntype{I}{X[-5,c]}
% Column type with uncertainty
\newcolumntype{U}{@{}X[-5,r]@{$\pm$}X[-5,l]@{}}
% Column type without uncertainty
\newcolumntype{C}{@{}X[-5,c]@{}}

\newcounter{papertableindexcounter}
\newcommand{\papertableindexheader}[0]{\multirow{2}{*}{\textsc{Index}}}
\newcommand{\papertableindex}[0]{\stepcounter{papertableindexcounter}
\arabic{papertableindexcounter}}
\newcommand{\papertableuheadersymbol}[1]{&\multicolumn{2}{c|[inner]}{$#1$}}
\newcommand{\papertableuheadersymbole}[1]{&\multicolumn{2}{c|[outer]}{$#1$}}
\newcommand{\papertableuheaderunit}[1]{&\multicolumn{2}{c|[inner]}{(#1)}}
\newcommand{\papertableuheaderunite}[1]{&\multicolumn{2}{c|[outer]}{(#1)}}
\newcommand{\papertablecheadersymbol}[1]{&$#1$}
\newcommand{\papertablecheaderunit}[2]{&($\pm$#1 #2)}

% Value in table with uncertainty.
\newcommand{\papertableuval}[2]{& #1 & #2}
% Value in table without uncertainty.
\newcommand{\papertablecval}[1]{& #1}

\newenvironment{papertable}[1]
{\setcounter{papertableindexcounter}{0} 
\begin{tabu} to \linewidth {#1}}
{\end{tabu}\vspace{12pt}}

\newcommand{\paperaxis}[9]
{title=#1,
axis x line = bottom,
xmin=#4,xmax=#6,
axis y line = left,
ymin=#5,ymax=#7,
height = 180pt,
grid=both,
x axis line style=-,
y axis line style=-,
x tick label style={
/pgf/number format/.cd,
fixed,
fixed zerofill,
precision=#8,
/tikz/.cd},
y tick label style={
/pgf/number format/.cd,
fixed,
fixed zerofill,
precision=#9,
/tikz/.cd}}
\newcommand{\paperaxisxlabel}[2]{
xlabel=\fontsize{10pt}{10pt}\selectfont#1$(#2)\rightarrow$}
\newcommand{\paperaxisylabel}[2]{
ylabel=\fontsize{10pt}{10pt}\selectfont#1$(#2)\rightarrow$}
\newcommand{\papergraphoutline}[4]{
\addplot [mark=none,line width=0.75pt] coordinates {
(#1,#2)
(#1,#4)
(#3,#4)
(#3,#2)
(#1,#2)};}

\newenvironment{papergraph}{
\begin{tikzpicture}
\begin{axis}}
{\end{axis}
\end{tikzpicture}}

\newcommand{\comment}[1]{}

\newcommand{\abs}[1]{\left\lvert#1\right\rvert}
\newcommand{\oo}[0]{\infty}
\newcommand{\sigmaSum}[3]{\sum\limits_{#1}^{#2} #3}
\newcommand{\limto}[3]{\lim\limits_{#1\rightarrow#2}#3}
\renewcommand{\d}[0]{\mathrm{d}}
\newcommand{\cross}[0]{\times}
\newcommand{\lp}{\left(}
\newcommand{\rp}{\right)}
\newcommand\pars[1]{\lp#1\rp}
\newcommand\sqbrack[1]{\left[#1\right]}
\newcommand\R{\mathbb{R}}
\newcommand\di{\partial}
\newcommand\x{\times}
\newcommand\del{\nabla}

\physics
\begin{document}
\papertitle{The Franck-Hertz Experiment}
\paperauth{A}{Khesin}{1002442029}
\paperauth{P}{Zavyalova}{1002345036}
\paperdate{March 21, 2018, Completed March 15, 2018}
\begin{paperabs}

	The Franck-Hertz experiment hints at the quantum nature of atoms by showing that they lose energy in discrete chunks when bombarded with electrons.
The energy they lose between two peaks was shown to be $(8.3\pm.3)\SI{E-19}{\joule}$. The associated wavelength of light that photons release was shown to be $(240\pm10)\si{\nano\meter}$.
	
\end{paperabs}

\begin{paper}
	
\papersec{Introduction}
	
	The Franck-Hertz experiment is well-known as the one of the first experiments to provide empirical evidence for the quantum nature of atoms. In the course of the experiment, one may demonstrate that electrons lose discrete amounts of energy, supporting Bohr's propositions about the nature of an atom. This is achieved by bombarding mercury atoms with a beam of electrons and observing the relationship between current and accelerating voltage of the electrons. This paper outlines the setup and procedure required for reproducing the experiment as well as results and observations. 
	
	Primary components of the experimental setup include a filament used for emitting electrons, electron-accelerating grids and an anode collecting the electrons; all of these are contained in a mercury vapor filled tube. Additionally, power supplies for heating the filament and accelerating the electrons, an oven for heating the tube, an electrometer and a voltmeter were used. The required devices are shown below.
	
	\paperfig{Setup}{\vspace{-1em} \pdf{setup} \vspace{-3em}}{Apparatus used in the experiment prior to wiring. The final apparatus was connected using the schematic circuit diagram further below. The panel supports four DC voltages with functionality explained below.}
	
	The circuit was setup according to the following diagram.

	\paperfig{Circuit}{\pdf{images}\vspace{-3em}}{Citcuit diagram used in the experiment. Ports E1 through E4 are localed on the device and the other nodes are on the accompanying box. Source for diagram: Serbanescu, 2017.}\\

	Prior to beginning data acquisition, the accelerating voltage was connected to the \textit{Manual} panel ports. The filament supply and the screen grid voltage were adjusted until the electrometer reflected smooth and regular raises and decreases as the accelerating voltage was slowly changed. 
	
	Throughout the course of the experiment, two voltages were recorded with the help of provided software: the accelerating voltage (X, channel 1) and the voltage recorded by the Keithley electrometer (Y, channel 2). All measurements were taken with the tube heated to \SI{169}{\celsius}.

\papersec{Observations}
	
	The resulting collected data took form of Channel 2 (Keithley elecrometer reading) vs. Channel 1 (accelerating potential) curves. A total of 17 curves were obtained, with some taking the desired smooth periodic form, and some having sudden peaks that could be explained with deviations of the tube temperature from the optimal value. The latter data sets could still be used whenever the horizontal distance between minima was not affected by the peaks. Presented below are three sample runs of varying smoothness and periodicity that correspond to the primary types of data collected throughout the experiment.  
	
	\paperfig{DataClean}{\pdf{16}}{Keithley electrometer potential vs. accelerating potential data for a run exhibiting a smooth, periodic curve with no spikes due to temperature fluctuations.}
	
	\paperfig{DataCleanNotSmooth}{\pdf{9}}{Keithley electrometer potential vs. accelerating potential data for a run exhibiting a periodic curve with no spikes due to temperature fluctuations. However, due to a consistently non-optimal temperature throughout this run, the curve is as smooth as desired, complicating locating the x-positions of the minima.}
	
	\paperfig{DataSpike}{\pdf{17}}{Keithley electrometer potential vs. accelerating potential data for a run exhibiting a smooth, periodic curve with potential spikes that may be attributed to temperature fluctuations throughout the run. Periodicity of the data still allows this curve to be used to measure horizontal distance between spikes.}
	
	The amount of energy transferred from an electron to a mercury atom in the vapor filling the tube in the course of an inelastic collision was inferred from the obtained curves by measuring the horizontal distances between the minima. The measured distances between peaks are shown below. \vspace{1em}
	
	\paperfig{Doublet}{\begin{papertable}{|[outer]I|[inner]C|[outer]}\paperoline
			\papertableindexheader&\textsc{Horizontal Distance Between Peaks (V)}\\
			\papertablecheadersymbol{(\pm.3)}\\\paperiline
			\papertableindex\papertablecval{5.2}\\\paperiline
			\papertableindex\papertablecval{5.1}\\\paperiline
			\papertableindex\papertablecval{5.4}\\\paperiline
			\papertableindex\papertablecval{5.4}\\\paperiline
			\papertableindex\papertablecval{5.1}\\\paperiline	
			\papertableindex\papertablecval{5.1}\\\paperiline
			\papertableindex\papertablecval{5.1}\\\paperiline
			\papertableindex\papertablecval{5.3}\\\paperiline
			\papertableindex\papertablecval{4.8}\\\paperiline
			\papertableindex\papertablecval{5.2}\\\paperiline
			\papertableindex\papertablecval{5.1}\\\paperiline
			\papertableindex\papertablecval{5.2}\\\paperiline
			\papertableindex\papertablecval{5.3}\\\paperiline
			\papertableindex\papertablecval{5.3}\\\paperiline
			\papertableindex\papertablecval{5.2}\\\paperiline
			\papertableindex\papertablecval{5.4}\\\paperiline
			\papertableindex\papertablecval{5.3}\\\paperoline
		\end{papertable} \vspace{-1em} }{Horizontal distance between minima of the Keithley electrometer potential vs. accelerating voltage for several runs of the Franck-Hertz experiment. }
	
	Since the distances were measured by selecting the position of the minima by using the software cursors, the uncertainty arises from the uncertainty of selection of the minimum point. Selecting the minimum point was made significantly more difficult due to the upward trend in the Channel 2 potential. 
	
\papersec{Analysis} 
	
The data in \figDoublet shows that the average potential difference between two peaks was $(5.2\pm.2)\si{\volt}$.
This was used to find the transfered energy.
The energy is related to the potential and the wavelength of the particles with the following equation.

\papereq{E}{E=eV=\frac{hc}\lambda}{}
\begin{paperwhere}
\papervar{E}{energy}{\joule}
\papervar{e}{fundamental charge}{\coulomb}
\papervar{V}{potential}{\volt}
\papervar{h}{Planck's constant}{\joule\second}
\papervar{c}{speed of light}{\meter\per\second}
\papervar{\lambda}{wavelength}{\meter}
\end{paperwhere}

The following values were used for fundamental constants. The value of $e$ was taken to be \SI{1.60217662E-19}{\coulomb}, the value of $h$ was taken to be \SI{6.62607004E-34}{\joule\second}, and the value of $c$ was taken to be \SI{299792458}{\meter\per\second}, the following results were obtained.

The Energy transfered was calculated with \eqE to be $(8.3\pm.3)\SI{E-19}{\joule}$.
The associated wavelength was calculated with \eqE to be $(240\pm10)\si{\nano\meter}$.

The original experiment used mercury instead of hydrogen gas as the gas needs to be heavy in order for the collisions with the electrons to be inelastic.
As hydrogen is much lighter than mercury, a collision of an electron with a hydrogen atom has a much lower chance of being inelastic.

On the graphs of the data plots, the patterns showed smooth dips and not sharp sawtooth patterns.
This was because the odds of an electron being absorbed by an atom increase at the corresponding potenials.
However, these odds are never a certainty, they slowly increase due to the various energies present in the electrons, and they reach their peak when the means of the two distributions, those being those of the energies of the mercury atoms and the electrons, are at the optimal distance apart. This made the results less precise as the minimum of the distribution was harder to locate and was possibly shifted, making this harder to measure than if it had been a sawtooth pattern.

\papersec{Conclusion}

Although this experiment was successful, it was not free from sources of error.
The most important source of error was the ability to accurately record and measure the data.
The software would frequently display large jumps and discontinuities as seen in \figDataSpike.
This made it difficult to get an accurate value for the distance between two consecutive minima.
This would have been reduced by collecting more data and aggregating all of the values.

The final results of this experiment hint at the quantum nature of atoms by showing that they lose energy in discrete chunks when bombarded with electrons.
The energy they lose between two peaks was shown to be $(8.3\pm.3)\SI{E-19}{\joule}$. The associated wavelength of light that photons release was shown to be $(240\pm10)\si{\nano\meter}$.
	
\papersec{Sources}

	\papersource{Serbanescu, R.,The Franck-Hertz Experiment, 2017}

\end{paper}
\end{document}