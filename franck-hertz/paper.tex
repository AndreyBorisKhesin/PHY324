\input{header}
\physics
\begin{document}
\papertitle{The Franck-Hertz Experiment}
\paperauth{A}{Khesin}{1002442029}
\paperauth{P}{Zavyalova}{1002345036}
\paperdate{March 121, 2018, Completed March 15, 2018}
\begin{paperabs}

	Abstract goes here
	
\end{paperabs}

\begin{paper}
	
\papersec{Introduction}
	
	The Franck-Hertz experiment is well-known as the one of the first experiments to provide empirical evidence for the quantum nature of atoms. In the course of the experiment, one may demonstrate that electrons lose discrete amounts of energy, supporting Bohr's propositions about the nature of an atom. This is achieved by bombarding mercury atoms with a beam of electrons and observing the relationship between current and accelerating voltage of the electrons. This paper outlines the setup and procedure required for reproducing the experiment as well as results and observations. 
	
	Primary components of the experimental setup include a filament used for emitting electrons, electron-accelerating grids and an anode collecting the electrons; all of these are contained in a mercury vapor filled tube. Additionally, power supplies for heating the filament and accelerating the electrons, an oven for heating the tube, an electrometer and a voltmeter were used. The required devices are shown below.
	
	\paperfig{Setup}{\vspace{-1em} \pdf{setup} \vspace{-3em}}{Apparatus used in the experiment prior to wiring. The final apparatus was connected using the schematic circuit diagram further below. The panel supports four DC voltages with functions explained below.}

\papersec{Observations}
	
	Observations go here
	
\papersec{Analysis} 
	
	Analysis goes here
	
\papersec{Conclusion}

	Conclusion goes here
	
\papersec{Sources}

	\papersource{Serbanescu, R.,The Franck-Hertz Experiment, 2017}

\end{paper}
\end{document}