\input{header}
\usepackage{tikz}
\usetikzlibrary{optics}

\usepackage{verbatim}

\physics
\begin{document}
\papertitle{Two-Terminal Devices}
\paperauth{A}{Khesin}{1002442029}
\paperauth{P}{Zavyalova}{1002345036}
\paperdate{March 19, 2018, Completed March 13, 2018}
\begin{paperabs}

	Abstract goes here
	
\end{paperabs}

\begin{paper}
	
\papersec{Introduction}
	
	The operation of a two-terminal device may be completely described by providing a relationship between the current through the device and the voltage across it. In this lab, the latter relation was determined for a variety of circuit elements, including resistors, a resistor in series with a dry cell, several diodes, a thermistor, and a section of a rectifier. 
	
	\textbf{TODO}: circuit diagram + description.
	
	Prior to wiring the circuit as displayed above, vertical and horizontal sensitivities of the oscilloscope along with its zeros were set. The desired I vs. V curves were displayed in the Y vs. X mode of the oscilloscope; to obtain current as opposed to potential on the dependent axes, the Y voltage was divided by \( R_2 = 4.7 \si{\ohm} \) before plotting. The final experimental setup looked as shown below.
	
	\paperfig{SetupMain}{\pdf{setup_main}}{Experimental setup used to determine I vs. V plots for a variety of provided circuit components, with the diode vacuum tube specifically investigated here.}
	
	To further interpret the obtained I vs. V curves, a multimeter was used to measure the resistance of each component, with the connections made in both possible directions. 
	
	Additionally, I vs. V graphs of several capacitors, inductors, and a potentiometer were obtained. The same circuit as in \textbf{TODO} was used, resulting in the following setup.
	
	\paperfig{SetupAdditional}{\pdf{setup_additional}}{Experimental setup used to determine I vs. V curves of additional circuit components, with a \( 100 \si{\kilo\ohm} \) connected here.}
	
\papersec{Observations}
	
	To obtain I vs. V plots for the circuit components investigated, measurements were first saved using the \textit{Save} function of the oscilloscope. Dividing the channel two voltage by the resistance as described above, the desired plots were obtained.  Note that \( R_1 \) in \textbf{TODO} varied for each circuit component. 
	
	% \begin{comment}
	\paperfig{ShortCircuit}{\pdf{char_curves/1-short}}{I vs. V curve for a short circuit, with \( R_1 \) set to \( 100 \si{\ohm} \).}
	\paperfig{Resistor47}{\pdf{char_curves/2-47}}{I vs. V curve for a \( 47 \si{\ohm} \) resistor, with \( R_1 \) set to \( 100 \si{\ohm} \).}	
	\paperfig{Resistor1000}{\pdf{char_curves/3-1000}}{I vs. V curve for a \( 1000 \si{\ohm} \) resistor, with \( R_1 \) set to \( 100 \si{\ohm} \).}	
	\paperfig{CellUp}{\pdf{char_curves/4a-cell-up}}{I vs. V curve for a \( 47 \si{\ohm} \) resistor in series with a 1 1/2 volt dry cell in the \textbf{positive direction???}. Here, \( R_1 \) of \( 100 \si{\ohm} \) was used.}	
	\paperfig{CellDown}{\pdf{char_curves/4b-cell-down}}{I vs. V curve for a \( 47 \si{\ohm} \) resistor in series with a 1 1/2 volt dry cell in the \textbf{negative direction???}. Here, \( R_1 \) of \( 100 \si{\ohm} \) was used.}
	\paperfig{Germanium}{\pdf{char_curves/5-germanium}}{I vs. V curve for a germanium diode. Here, \( R_1 \) of \( 100 \si{\ohm} \) was used.}
	\paperfig{Silicon}{\pdf{char_curves/6-silicon}}{I vs. V curve for a silicon diode. Here, \( R_1 \) of \( 100 \si{\ohm} \) was used.}
	\paperfig{Selenium}{\pdf{char_curves/7-selenium}}{I vs. V curve for a one section of a selenium rectifier. Here, \( R_1 \) of \( 100 \si{\ohm} \) was used.}
	\paperfig{Vacuum}{\pdf{char_curves/8-vacuum}}{I vs. V curve for a diode vacuum tube. Here, \( R_1 \) of \( 0 \si{\ohm} \) was used.}
	\paperfig{Zener}{\pdf{char_curves/9-zener}}{I vs. V curve for a zener diode. Here, \( R_1 \) of \( 100 \si{\ohm} \) was used.}
	\paperfig{Thermistor}{\pdf{char_curves/10-thermistor}}{I vs. V curve for a thermistor. Here, \( R_1 \) of \( 50 \si{\ohm} \) was used.}
	% \end{comment}
	
	\begin{comment}
	\paperfig{ShortCircuit}{\pdf{char_curves\\1-short}}{I vs. V curve for a short circuit, with \( R_1 \) set to \( 100 \si{\ohm} \).}
	\paperfig{ShortCircuit}{\pdf{char_curves\\2-47}}{I vs. V curve for a \( 47 \si{\ohm} \) resistor, with \( R_1 \) set to \( 100 \si{\ohm} \).}
	\paperfig{Resistor1000}{\pdf{char_curves\\3-1000}}{I vs. V curve for a \( 1000 \si{\ohm} \) resistor, with \( R_1 \) set to \( 100 \si{\ohm} \).}	
	\paperfig{CellUp}{\pdf{char_curves\\4a-cell-up}}{I vs. V curve for a \( 47 \si{\ohm} \) resistor in series with a 1 1/2 volt dry cell in the \textbf{positive direction???}. Here, \( R_1 \) of \( 100 \si{\ohm} \) was used.}
	\paperfig{CellDown}{\pdf{char_curves\\4b-cell-down}}{I vs. V curve for a \( 47 \si{\ohm} \) resistor in series with a 1 1/2 volt dry cell in the \textbf{negative direction???}. Here, \( R_1 \) of \( 100 \si{\ohm} \) was used.}
	\paperfig{Germanium}{\pdf{char_curves\\5-germanium}}{I vs. V curve for a germanium diode. Here, \( R_1 \) of \( 100 \si{\ohm} \) was used.}
	\paperfig{Silicon}{\pdf{char_curves\\6-silicon}}{I vs. V curve for a silicon diode. Here, \( R_1 \) of \( 100 \si{\ohm} \) was used.}
	paperfig{Selenium}{\pdf{char_curves\\7-selenium}}{I vs. V curve for a one section of a selenium rectifier. Here, \( R_1 \) of \( 100 \si{\ohm} \) was used.}
	\paperfig{Vacuum}{\pdf{char_curves\\8-vacuum}}{I vs. V curve for a diode vacuum tube. Here, \( R_1 \) of \( 0 \si{\ohm} \) was used.}
	\paperfig{Zener}{\pdf{char_curves\\9-zener}}{I vs. V curve for a zener diode. Here, \( R_1 \) of \( 100 \si{\ohm} \) was used.}
	\paperfig{Thermistor}{\pdf{char_curves\\10-thermistor}}{I vs. V curve for a thermistor. Here, \( R_1 \) of \( 50 \si{\ohm} \) was used.}
	\end{comment}
	
\papersec{Analysis} 
	
	
	
\papersec{Conclusion}

	Conclusion goes here
	
\papersec{Sources}

	\papersource{J.V., R.M.S., Two-terminal devices, 2010}

\end{paper}
\end{document}