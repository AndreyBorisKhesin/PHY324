\input{header}
\physics
\begin{document}
\papertitle{Interferometers}
\paperauth{A}{Khesin}{1002442029}
\paperauth{P}{Zavyalova}{1002345036}
\paperdate{}
\begin{paperabs}
	
	Abstract goes here
	
\end{paperabs}
	
\begin{paper}
	
\papersec{Introduction}
	
	Interference patterns produced by various light sources may be studied using interferometers, which operate by dividing either the wavefront or the amplitude of the original light beam. In this experiment, the Michelson interferometer was used extensively to determine the indexes of refraction of a transparent solid and a gas after being calibrated using a sodium light. As seen from the schematic below, the Michelson interferometer operates by splitting the amplitude of the incident beam. 
	
	\paperfig{MichelsonInterferometer}{% Still need to figure out labels (esp. the slide location)
	\begin{tikzpicture}[use optics]
		\node[thick optics element] at (-2.5, 0) {};
		\node[thick optics element, rotate = -45] at (0, 0) {};
		\node[generic optics io, io body aspect ratio = 1] at (-4.0, 0) {};
		\node[mirror] at (3.0, 0) {};
		\node[mirror, rotate = 90] at (0, 3.0) {};
		\node[thick optics element, rotate = -45] at (1.5, -0.5) {};
		\draw[->-] (-3.38, 0) -- (-0.15, 0);
		\draw[->-] (-0.15, 0) -- (-0.15, 3.0);
		\draw[->-] (0, 3.0) -- (0, 0.15);
		\draw[] (0, 0.15) -- (0.15, 0);
		\draw[->-] (0.15, 0) -- (0.15, -3.0);
		\draw[] (-0.15, 0) -- (0, -0.15);
		\draw[->-] (0, -0.15) -- (1.7, -0.15);
		\draw[] (1.7, -0.15) -- (1.85, -0.3);
		\draw[->-] (1.85, -0.3) -- (3.0, -0.3);
		\draw[->-] (3.0, -0.45) -- (1.70, -0.45);
		\draw[] (1.70, -0.45) -- (1.55, -0.3);
		\draw[->-] (1.55, -0.3) -- (-0.15, -0.3);
		\draw[->-] (-0.15, -0.3) -- (-0.15, -3.0);
		\node[thin optics element, rotate = -90] at (0, -3.0) {};
		\node[thin optics element, rotate = 90] at (0, 2.25) {};
	\end{tikzpicture}}{\textbf{TODO}}
	
	Prior to making any measurements, the Michelson interferometer was calibrated against a sodium lamp with an assumed mean wavelength value of \( \lambda = 598.3 \) \si{nm}. The calibration curve itself was obtained by moving the carriage toward the observer, counting the number of fringes disappearing from the middle of the field of view and plotting the results. 
	
	Once the interferometer was calibrated, the setup could be used to determine the index of refraction of various materials. For instance, the index of refraction of a transparent solid---a microscope slide---was found. In general, the index of refraction of a solid can be shown to obey the following relation.
	
	\papereq{SolidRefractiveIndex}{\mu = \frac{m \lambda}{2d} + 1}{\textbf{TODO}}
	\begin{paperwhere}
		\papervar{\mu}{index of refraction}{}
		\papervar{m}{number of fringes displaced}{}
		\papervar{\lambda}{wavelength of light}{\meter}
		\papervar{d}{thickness of the solid}{\meter}
	\end{paperwhere}
	
	As seen from \eqSolidRefractiveIndex, inserting a transparent solid into the optical path of the beam results in a displacement of the interference pattern produced by the light source. 
	
\papersec{Observations}
	
	Observations go here
	
\papersec{Analysis} 
	
	Analysis goes here
	
\papersec{Conclusion}

	Conclusion goes here
	
\papersec{Sources}

	\papersource{}

\end{paper}
\end{document}